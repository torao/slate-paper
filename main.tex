\documentclass[xelatex,ja=standard,a4paper,11pt]{bxjsarticle}

\usepackage{geometry}
\usepackage{titlesec}
\usepackage{setspace}
\usepackage{hyperref}
\usepackage{indentfirst}
\usepackage{amsmath}
\usepackage{amssymb}
\usepackage{amsfonts}
\usepackage{amsthm}
\usepackage{booktabs}
\usepackage[dvipdfmx]{graphicx}
\usepackage[tableposition=bottom]{caption}
\setstretch{1.2}
\newtheorem{definition}{定義}[section]
\newtheorem{proposition}{命題}[section]
\newtheorem{theorem}{定理}[section]
\newtheorem{lemma}{補題}[section]

% draft-mark on each page
\pagestyle{myheadings}
\usepackage[useregional]{datetime2}
\DTMsetdatestyle{iso}
\markright{{\sf \scriptsize DRAFT @ \DTMnow}}

\title{Slate: 分散ログの分岐検出のための\\追記最適化マークルツリー}
% \title{Slate: An Append-Optimized Merkle Tree for Fork Detection in Distributed Logs }
\author{鷹見 虎男}
\date{\today}
\hypersetup{
  pdftitle    = {Slate: 分散ログの分岐検出のための追記最適化マークルツリー},
  pdfauthor   = {鷹見 虎男},
  pdfkeywords = {Slate, Stratified Hash Tree, Merkle Tree, Hash Tree, Distributed Log, Append-Optimized, Temporal Optimized, Synchronization, Fork Detection, Difference Detection}
}

\begin{document}
\maketitle

\begin{abstract}
大規模分散システムにおいて、分散するレプリカ間でのトランザクションログの迅速な修復と同期は、可用性・整合性・障害回復の性能に大きく影響する。現実的な分散ログは追記偏重のアクセス特性を持ち、その同期性能は分岐 (フォーク) 位置の検出と修復の効率に大きく依存する。一方で、従来の方法はそのようなログの末尾からの線形探索で分岐を検出するため大規模なデータセットに対して非効率であり、適用可能な局面は限定的である。本論文は、大きな履歴遡及が頻繁に発生する状況でも効率的に機能する、追記最適化と時間局所性に基づくアクセス特性を持つ Merkle Tree である Slate(Stratified Hash Tree)を提案する。本論文ではまた、Slate において最新エントリから任意の位置のデータに到達するまでの I/O 回数分布がハミング距離に従うことを示し、分布の傾向により I/O が最新データの近傍で削減されることを理論化する。これに基づき、I/O コストが性能に支配的な現実的な状況において、$O(\log n)$ 回の I/O での分岐 (差異) 位置の同定と $O(n)$ 回の I/O での同期転送を満たす、暗号論的に確立された安全性に基づく修復手順を与える。さらに、証明サイズと比較ラウンド数においての上界を導き、追記偏重のワークロードでの現実的な期待コストが最新データからのハミング距離に支配される非対称な性能特性を示す。

Rust の実装を用いた評価では、従来の Merkle Tree 構造に比べ、最新付近に偏る差分同期で転送量・探索 I/O・キャッシュミスが一貫して低減することを確認した。Slate は分散トランザクションログ、非同期レプリケーションログ、ブロックチェーン、および Git に類するハッシュグラフの分岐検出と修復に適用可能であり、追記最適化・時間局所性・認証性を両立する実用的基盤を提供する。
\end{abstract}

\textbf{Keywords}: Slate, Stratified Hash Tree, Merkle Tree, Hash Tree, Distributed Log, Append-Optimized, Temporal Optimized, Synchronization, Fork Detection, Difference Detection

% \tableofcontents
% \newpage

\section{INTRODUCTION}

\subsection{動機}

多くの大規模分散システムにおける可用性・整合性・障害回復の性能は、レプリカ間の順序付き更新ログの複製を基礎としている。分散ログは現実的な多くのワークロードで追記偏重 (append-heavy) であり\cite{kafka2011}、さらに最新近傍に参照が偏る時間的局所性を持つ。ログの末尾存在する確定していないデータ (未コミットのデータ) は、レプリカ間の自然な遅延やネットワーク分断、一時的な故障、構成変更により、複数のレプリカで異なる (分岐する) 可能性がある。このような状況において、レプリカ間の各ログの分岐を検出し、修復する効率が、システムの性能や復旧時間 (RTO/RPO) に直接影響する。

Merkle Tree \cite{merkle1989} またはハッシュツリーは順序付き集合を暗号論的な厳密さの下で効率的に比較・検証することができるデータ構造である。この検証を部分木へ再帰的に適用することで、差異の有無のみならず、順序付き集合のどこが異なるかを効率的に特定することができる。Merkle Tree の亜種である History Tree \cite{crosby2009} は、追記専用を前提とした動的な構造を持ち、第三者が発行済みの証明書を監査するための Certificate Transparency (CT) \cite{ct2021} として現実のシステムで稼働している。

ログを History Tree 構造に適用することで効率的な分岐検出を行うことができるが、History Tree は証明書のような信頼性の高いデータの改ざん検出や発行証跡として使用することに焦点を当てており、この構造を追記最適化する試みや、ログの分岐検出やその修復を行う議論は我々の知る限り行われていない。

本研究は History Tree を分散ログの分岐検出と修復の用途に拡張し、二次記憶装置に対して追記最適化された直列化構造と、最新に近いエントリをより効率的に参照できる時間局所性 (temporal locality) を持つデータ構造である \textbf{Slate} (Stratified Hash Tree) を提案する。そして、本論文では Slate において最新エントリから任意の葉に到達するまでの I/O 回数がハミング距離に従うことを示し、その上限を示すことで、最新近傍において I/O 回数 (キャッシュミス回数) が優先的に削減されることを示す。

\subsection{問題設定}

本論文は単調増加するインデックス $i\in\mathbb{N}$ によって識別される値の列であるログ $D=(v_1,v_2,\ldots,v_n)$ を想定する。ログは追記 ($\mathrm{append}$) と切り詰め ($\mathrm{truncate}$) のみによって変更され、個別の値が更新されることはない。2 つの非ビザンチンレプリカ $A$ と $B$  が保持するそれぞれのログを $D^{(A)}$ と $D^{(B)}$ とし、両者の\textbf{最長共通接頭辞長} (longest common prefix; LCP) を $d=\mathrm{LCP}(D^{(A)},D^{(B)})$ と定義する。本文の目的は次の二点である:

\begin{enumerate}
    \item \textbf{インデックス参照} (index reference): 想定するアクセスモデルは、\textbf{時間局所性} (temporal locality) に基づき、最新に近い値に統計的に偏ると仮定する。形式的には \(\delta=n-i\) に対して \(\mathrm{Pr}[\mathrm{get}(i)]=f(\delta)\) が \(\delta\) の単調減少関数となるケースである (例えば幾何分布、Zipf 近似、指数減衰など)。
    \item \textbf{分岐点検出} (fork detection): レプリカ間のインタラクション数と転送量を最小化して、2 つのログ $D^{(A)}$ と $D^{(B)}$ の LCP 長 $d$ を特定する。言い換えると、$D^{(A)}$ と $D^{(B)}$ で異なる値が最初に現われる位置 $d+1$ を特定する。
    \item \textbf{修復} (repair): 検出した $d$ を用いて、片方のログの接尾辞をもう片方のログに転送して整合 (同期) する。これは、例えば $D_A$ を $(v_1,v_2,\ldots,v_d)$ に切り詰め、他方の $D_B$ の接尾辞 $(v_{d+1},\ldots)$ を転送して追加する。
\end{enumerate}

これらは、追記偏重の大規模ログ運用下で分岐と再同期が頻発する環境において、最小のインタラクションと I/O で LCP の特定と修復を実現することを目的とする。

各値 $v_i$ に適用可能な暗号学的ハッシュ関数 $h$ が定義されている。本論文では、セキュリティパラメータ \(\lambda\) に対して計算量的に原像困難性、第二原像困難性、衝突困難性を満たし、任意の多項式時間攻撃者の成功確率が $\textsf{negl}(\lambda)$ であるような、標準的なハッシュ関数族 $H_\lambda:\{0,1\}^*\to\{0,1\}^\lambda$ を想定する。\(H\) に属するドメイン分離されたハッシュ関数を \(h_A,h_B,\ldots\) のように表記する。 

この設定のもと、我々は(1)\textbf{最新から過去への探索を対数段数で保証}し、(2)\textbf{距離の確率分布が二項分布に従うこと}を示し、(3)\textbf{分岐検出・部分転送・修復}の各手順に対する\textbf{計算/通信上界}を与える。特に、追記偏重ワークロードでは差分が最新近傍に集中するため、期待コストが\textbf{最新側のハミング距離}で支配される\emph{非対称}な性能特性が得られる。

\subsection{本研究の貢献}
本研究は、追記最適化と時間局所性を持ち、効率的な分岐検出 (差異位置検出) が可能なデータ構造である \textbf{Slate} (Stratified Hash Tree) を提案する。Slate は以下に述べる技術的な貢献により、既存の Merkle Tree およびその亜種である History Tree の手法を拡張する:

\begin{enumerate}
  \item \textbf{追記最適化ハッシュツリーの導入}: 従来の Merkle Tree が静的なデータセットに対する完全性検証に特化していたのに対し、本研究では追記操作に最適化された動的なログ構造への拡張を実現した。これにより、History Tree と類似する成長パターンを持ちながら、追記に適した構成を明確に定義し、効率的な差分検出と直列化を可能にする新しいハッシュツリーの構造を確立した。
  \item \textbf{時間局所性に基づくアクセス効率の最適化}: ハミング距離の上限 $c^+(x)\sim\log (n-x)$ と下限 $c^-(x)\sim h-\log x$ を数学的に定義し、この境界に従い、最新近傍で I/O 効率が高く、最悪ケースでも通常の二分木探索と同等な木構造を開発した。この結果、最新データへのアクセスが $O(1)$ で、最新から離れるにつれて $O(\log(n-x))$ で増加し、最悪ケースでも $O(\log n)$ 回の I/O 操作で完了する非対称アクセス特性を開発し、ログの分岐点の特定を従来の $O(n)$ から $O(\log n)$ に改善した。
  \item \textbf{部分永続的データ構造の効率的なエンコード}: 部分永続化の効率的なバージョン階層を 2 進数展開として自然にエンコードする手法を確立し、Driscoll ら \cite{driscoll1986} の一時的な永続化手法で必要とされていた複雑なポインタ管理やバージョンスタンプを排除した。各世代 $i$ で追加されるノードを単一のエントリ $e_i$ に集約ことで、シンプルかつ効率的な部分永続化を実現した。
  \item \textbf{並行アクセスの安全性保証}: 永続的データ構造の特徴を利用し、SWMR (Single-Write/Multi-Read) 並行実行モデルに基づく安全な並行アクセスを可能にした。これは、スナップショット一貫性により、複数の読み取り操作が単一の追記操作と競合なく並行実行できることを保証する (切り詰め操作を採用する場合でも、コミット指標の保護領域では並行実行可能であることを示す)。
  \item \textbf{決定論的な木構造の構築アルゴリズム}: 葉ノード数 $n$ の 2 進数展開 $n=\sum_{h\ge 0}b_h 2^h$ に基づく決定論的な構築手法を確立し、同一データに対して常に同一の木構造が生成されることを保証した。
  \item \textbf{効率的な直列化形式}: 追記操作が単一のファイル追記で完結する直列化形式を設計し、二次記憶装置への I/O を最小化した。空間複雑性を $\Theta(n\log n)$ におさえながら、実用的な性能を達成した。

\end{enumerate}
% \textbf{分散同期の計算量改善} RaftのログレプリケーションやGitの分岐点検出において、従来のO(n)線形探索をO(log n)に改善する理論的基盤を提供した。 
% \textbf{実装効率を考慮したビット演算の活用} popcount、clz、ctzなどのCPU命令を活用した効率的な実装手法を体系化し、理論と実装のギャップを最小化した。 
% \textbf{階層的なデータ管理の新しいパラダイム} LSM-TreeやStratified B-Treeとは異なる、時間的局所性を空間的局所性に変換する新しい階層化手法を確立した。 
% \textbf{トランザクションログへの実用的な適用} WAL(Write-Ahead Logging)、ブロックチェーン、分散バージョン管理システムなど、実用的なシステムへの適用可能性を示した。 

これらの貢献により、 本研究は従来の Merkle Tree および History Tree を拡張し、理論的な優位性と実装の実用性を両立する新しいデータ構造を確立した。これは分散システムにおける効率的なログ同期と検証に適用可能な新しい基盤技術を提供する。


\section{Preliminaries}

\subsection{Merkle Tree とその亜種}

典型的な Merkle Tree \cite{merkle1989} は完全二分木であり、すべてのノードはそのノードを根とする部分木のダイジェストを表すハッシュ値を持つ。各葉ノードは葉に含まれるデータ要素 $d_i$ のハッシュ値 $H(d_i)$ をハッシュ値として持つ。各中間ノード $b_{i,j}$ は 2 つの子ノードのハッシュ値の連結から計算されたハッシュ値 $H(l.hash\,||\,r.hash)$ を持つ。各要素に対する包含証明 (inclusion proof) を生成するための時間計算量および空間計算量は共に $O(\log n)$ である。

\textbf{History Tree} \cite{crosby2009} は要素の動的な追加操作をサポートするように拡張した Merkle Tree の亜種である。この木構造は 1 つ以上の完全二分部分木と、それらを連結するノードで構成されている。ここで最も高い完全二分部分木を最左に、高さの降順で左から右に並んでいるものとする。History Tree に追加される新しいデータは、まず木構造の最も右の葉ノードとして配置され、これを高さ $h=0$ の完全二分部分木とみなす。次に、左の 2 つの木構造を接続する中間ノードを再帰的に構築する。この過程で、複数の完全二分部分木が統合され 1 つの完全二分部分木となることがある。この追加操作は $O(\log n)$ で完了する。

Slate は標準的な Merkle Tree に対して以下の 3 点を拡張する:

\begin{enumerate}
    \item \textbf{累積的成長}: 既存の完全二分部分木を維持しながら木を生長させ、過去の任意時点の構造を再現可能にする。
    \item \textbf{層様配置}: 同一世代に生成されるノードを単一のエントリに集約し、時間的局所性を空間的局所性に変換する。
    \item \textbf{非対称アクセス効率}: 右枝方向の探索が同一エントリ内で完結するため、新しいデータへのアクセスが効率化される傾向を持つ。
\end{enumerate}

\subsection{Hash Function}

Slate では Merkle Tree における第二原像攻撃を防止する目的でノードの高さ $j$ ごとにドメイン分離したハッシュ関数を使用する。本論文では原像耐性、第二原像耐性、衝突耐性を持つ一般的な暗号論的ハッシュ関数 $\mathcal{H}:\{0,1\}^*\to\{0,1\}^\lambda$ を仮定する。ここで $\lambda$ はハッシュ出力のビット長である。Slate では $\mathcal{H}$ からドメイン $j$ で分離されたハッシュ関数 $h_j$ を使用する。これは、具体的には、高さ $j$ を入力の接頭辞として用いたハッシュ関数 $h_j(m) = \mathcal{H}(j\,||\,m)$ を使用できる。ここで $||$ はビット列の連結演算子を表している。

\subsection{Log Structure}

本論文では、すべてのデータが時系列順に末尾へのみ追記され、一度記録されたデータは不変となる追記専用 (append-only) のデータ構造を\textbf{ログ構造} (log structure) とする。$\mathbb{N}$ を自然数全体の集合 $\{1,2,\ldots\}$、$\mathbb{N}_0$ を非負値 $\mathbb{N} \cup \{0\}$ としたとき、離散インデックス $i \in \mathbb{N}$ と値集合 $V$ に対し、ログ (log) $L$ は長さ $n \in \mathbb{N}_0$ を持つ有限列 (\ref{eq:log-structure}) として定義する。
\begin{equation}
    L = \langle x_1, x_2, \ldots, x_n \rangle, \quad x_i \in V, i \in \mathbb{N}
    \label{eq:log-structure}
\end{equation}
単なる順序付きデータ列とは異なり、ログ構造では既存のデータの変更や任意の位置への挿入・削除を許さないことを示唆しており、履歴の完全性と監査可能性が構造的に保証される。

\subsection{Temporal Locality}

時間的局所性 (temporal locality) とは、時刻 $t$ にアクセスされたデータ要素が時刻 $t+\Delta t$ においても高い確率でアクセスされる性質を指す。例えば Web リクエストにおいては時間経過に対し重尾 (tail-heavy) な対数正規分布なることが知られており \cite{almeida1996, barford1998}、ブロックチェーンのような append-only のデータ構造においても最近追加されたデータが次にアクセスされる要素である確率が重尾 (heavy-tailed) な分布、特にパレート分布やべき乗則に従うことが知られている \cite{sornette2025}。これは、最近追加されたデータを優先的にキャッシュするデータ構造を用いることでキャッシュ効率や I/O 性能を改善できることを示唆している。

\subsection{Hamming distance}

2 つの $n$ ビット列 $x,y \in \{0,1\}^n$ のハミング距離 (Hamming distance) $d_H(x,y)$ は異なるビットの数として定義される。これは式 (\ref{eq:hamming_distance}) に示すように XOR 演算と \textrm{popcount} CPU 命令により効率的に計算できる。

\begin{equation}
d_H(x,y) = |\{i: x_i \ne y_i\}| = \mathrm{popcount}(x \oplus y)
\label{eq:hamming_distance}
\end{equation}

完全二分木に含まれる 2 つの葉のインデックス $i,j \in \{0,1,\ldots\}$ のハミング距離 $d_H$ は、それらの葉ノードの最小共通祖先 (lowest common ancestor) の高さを決定する。具体的には、最小共通祖先はルートから深さ $\lfloor \log_2 (i \oplus j) \rfloor$ に位置する。例えば、インデックス $i=5=101_2$ と $j=7=111_2$ の場合、$i \oplus j = 010_2$ より、最小共通祖先は深さ 1 にあり、両方の葉ノードは高さ 2 の部分木内に存在する。


\section{Slate: Structure and Algorithm}

本セクションでは Slate の構造的特徴と基本操作を形式的に定義する。Slate は History Tree に基づいており、最近追加されたデータとそれに関連するノードが局所化され末尾に追記される直列化形式に特徴を持つ。木構造はその多くの領域が完全二分部分木 (PSBT) で構成されている。最近追加された新しいデータは上層の小さな PSBT に配置され、古いデータは徐々に下層の PSBT にマージされる。この構造により、append 操作に $O(1)$ の I/O、get 操作にはハミング距離に応じた幾何分布的な I/O コストを持つようになる。

以下、データ構造の定義(3.1節)、基本操作(3.2節)、計算複雑度の解析(3.3節)、正しさの証明(3.4節)の順に説明する。

\subsection{Data Structure}

Slate は History Tree \cite{crosby2009} に基づいた追記操作が可能な Merkle Tree \cite{merkle1989} であり、その葉ノードをそれぞれ 1 つのデータに対応させることで順序付きのデータ列を表現する。本論文では $n$ 個の葉ノードを持つ木構造を $T_n$ と表記し、特に $T_n$ の成長過程に注目するとき $n$-世代の木と表現する。ここで $T_0 = \emptyset$ である。便宜上、あるノード $b$ が木構造 $T$ に含まれていることを $b \in T$ と表記し、$T_1$ が $T_2$ の部分木であることを $T_1 \subseteq T_2$ と表記する。

\textbf{添字}: $i \in \mathbb{N}$ を 1 から始まる自然数とし、$i-1$ 世代の木構造 $T_{i-1}$ に $i$ 番目のデータ $d_i$ を追加 (統合) するために発生するノードを $b_{i,j}$ とする。$j$ は $b_{i,j}$ をルートとする部分木の中で最も\textit{遠い}葉ノードまでのホップ数、つまり $b_{i,j}$ の高さを表している。$i$ と $j$ はそれぞれ式 (\ref{eq:ij}) のように表すことができる。
\begin{equation}
i \in \{1,2,\ldots\}: 0 \le j \le \left\lceil \log_2 i \right\rceil
\label{eq:ij}
\end{equation}
$j=0$ は葉ノードのケースであり、$b_{i,0}=b_i$ と表す。また、$T_n$ に含まれる任意のノード $b_{i,j}$ をルートとしてすべての子孫ノードを含む部分木を $T_{i,j} \subseteq T_n$ と表す。

\textbf{完全二分木}: 本論文は木構造が完全二分木 (perfect binary tree) であることを強調するために $'$ 記号を用いる。$T'_n$ は $n$ 世代目の木構造が完全二分木であることを意味する。また $T'_{i,j}$ は部分木 $T_{i,j}$ が完全二分木であることを意味し、このとき $T'_{i,j}$ のルートを $b'_{i,j}$ と表す。

\subsubsection{部分的永続データ構造}
Slate の木は関数型プログラミングの文脈で部分的永続データ構造 \cite{driscoll1986} (partially persisten data structure) である。任意の完全二分部分木のルートノード $b'_{i,j} \in T_n$ は\textbf{永続性} (persistent) を持つ。つまり $n$ 以降のすべての世代の木 $T_m$ においても同じノード $b'_{i,j}$ が存在し続ける。ノード $b_{i,j}$ をルートとする部分木が完全二分木ではない場合、$b_{i,j}$ はその世代に限り存在する\textbf{一過性} (ephemeral) であり、それ以降の世代では木構造に現われることはない。

$T_n$ に含まれる任意のノード $b_{i,j}$ が永続性ノードかどうか、言い換えると、$b_{i,j}$ をルートとする部分木が完全二分木かどうかは式 (\ref{eq:persistent_node}) で評価することができる。
\begin{equation}
    \begin{cases}
        i \bmod{2^j} = 0 & \text{if $b_{i,j}$ is persistent} \\
        i \bmod{2^j} \ne 0 & \text{otherwise, it is ephemeral}
    \end{cases}
    \label{eq:persistent_node}
\end{equation}
すべての葉ノード $b_{i,0}$ は $i \bmod{2^0} = 0$ であることから永続性ノードであることは明らかである。

式 (\ref{eq:persistent_node}) に示す判定は、Slate が葉ノードを $2^j$ 単位のグループに分割する階層構造を持つことに基づいている。つまり、$T_n$ の葉ノードを $2^j$ ごとのグループに分割し、グループ内のすべての葉ノードが埋まっているグループは、対応する部分木が完全二分木であることを意味する。例えば、高さ $j=2$ では容量 $4$ のグループとなり、$n=11$ に対して 2 つのグループが完全に埋まり、1 つのグループが部分的に埋まっている状態となる。

木構造 $T_n$ は、互いに独立した (部分構造の関係にない) 部分木の中で最大と見なされる完全二分木の集合で構成されている。本論文では、そのような完全二分部分木を\textbf{層別木} (stratum tree) と呼び、左から $s \in [1,\mathrm{popcount}(n)]$ 番目に位置する層別木を $T'^{(s)}_{i_s,j_s}$ と表す。 

\subsubsection{エントリ}
木構造への追記操作 $T_{i-1} \overset{d_i}{\to} T_i$ の過程で新規に計算されるノードの集合を\textbf{エントリ} (entry) と定義し $e_i$ で表す。$e_i$ に含まれるすべてのノードの添字 $i$ (世代番号) はすべて同じであることに注意。ここで $e_i$ を構成するノードの添字 $j$ を求める。

$i$ の 2 進数表現に対し、最右桁から連続する $0$ の個数を得る関数を $\mathrm{ctz}(i)$ とする。言い換えると $\mathrm{ctz}(i)$ は $i$ の最も右に位置する $1$ の位置を意味している。$i$ の 2 進数表現において、$0$ から $\mathrm{ctz}(i)$ 桁は永続性ノードを表しており、$\mathrm{ctz}(i) + 1$ 桁以降に現われる $1$ は一過性ノードを表している。それぞれ $j$ の集合に変換すると式 (\ref{eq:ejp}) と (\ref{eq:eje}) のようになる。
\begin{eqnarray}
    \mathcal{J}'(i) & = & \{\mathrm{ctz}(i), \mathrm{ctz}(i)-1, \ldots, 0\} \label{eq:ejp} \\
    \mathcal{J}(i) & = & \left\{ p + 1 \,\left|\, p \in \mathbb{N}_0, \left\lfloor \frac{i}{2^p} \right\rfloor \bmod 2 = 1, p > \mathrm{ctz}(i) \right.\right\} \label{eq:eje}
\end{eqnarray}
$\mathcal{J}'$ と $\mathcal{J}$ はそれぞれ永続性ノードと一過性ノードを区別している。これより、エントリ $e_i$ と $e_i$ が含むノード数 $|e_i|$ は式 (\ref{eq:eij}) と (\ref{eq:ei_size}) のように表すことができる。
\begin{eqnarray}
    e_i & = & \{ b'_{i,j} \mid j \in \mathcal{J}'(i) \} \cup \{b_{i,j} \mid j \in \mathcal{J}(i)\}
    \label{eq:eij} \\
    |e_i| & = & |\mathcal{J}'(i)| + |\mathcal{J}(i)| = \mathrm{ctz}(i) + \mathrm{popcount}(i)
    \label{eq:ei_size}
\end{eqnarray}
$e_n$ の空間計算量および構築にかかる時間計算量は共に $O(\log_2 n)$ である。

後述するように、エントリは直列化形式においての保存単位であり、時間的局所性を空間的局所性に変換する、本論文において重要なパーツである。

\subsubsection{木構造の構築}
Slate は、それぞれの層別木を一世代限りの一過性ノードで接続して決定論的な木を構築する。図 \ref{fig:t22} は $n=22$ の木構造 $T_{22}$ を表しており、3 つの層別木 $T'_{16,4}, T'_{20,2}, T'_{22,1}$ を 2 つの一過性ノード $b_{22,5},b_{22,3}$ で接続して木を成している様子を表している。

$T_n$ に含まれる層別木の数は $n$ の 2 進数表現での 1 の個数 $\mathrm{popcount}(n)$ に等しく、ぞれぞれの層別木の高さ $j$ は、$n$ の 2 進数表現において対応する 1 が現われる位置と等しい (最右桁の位置を 0 とする)。高さ $j$ の完全二分木は $2^j$ 個の葉ノードを含んでいることから、$n$ は式 (\ref{eq:n}) のようにそれぞれの層別木の葉ノード数の和で表すこともできる。
\begin{equation}
n = \sum_{j \ge 0} \alpha_j\,2^j \quad (\alpha_j \in \{0,1\})
\label{eq:n}
\end{equation}
ここで $\alpha_j$ は $n$ の 2 進数表現の $j$ 桁目の値を意味する。これより $T_n$ に含まれる各層別木のルートノードの集合 $\mathcal{B}'_n$ は式 (\ref{eq:pbst_roots}) のように表すことができる。
\begin{equation}
\mathcal{B}'_n = \left\{b'_{i,j}\,\left|\, j \in \mathbb{N}_0,  
  \left\lfloor\frac{n}{2^j}\right\rfloor \bmod{2} = 1,i=\left\lfloor\frac{n}{2^j}\right\rfloor\cdot2^j \right.\right\} 
\label{eq:pbst_roots}
\end{equation}
$\mathcal{B}'_n$ が 1 つのノードのみを含む場合、そのノードは $b'_{n,\log_2 n}$ でなければならず、木構造全体が完全二分木 $T'_n$ であることを示している。そうでない場合、複数の層別木を結合する一過性ノードを生成する必要がある。二項演算子 $\otimes$ を左右のオペランドを子に持つ新しいノードを生成する演算としたとき、降順に並べた $\mathcal{B}'_n$ に対する右畳み込み操作 $\mathrm{reduce\_right}(\otimes, (\mathcal{B}'_n)_{\downarrow})$ は一過性ノードを生成しながら最終的に $T_n$ のルートノードを生成する。

\begin{proposition}
    \label{prop:bin} 木構造 $T_n$ に含まれる層別木は、$n$ の 2 進数表現における数値 1 に対応しており、その 1 が出現する位置を高さとする。 \label{prop:bit_psbt}
\end{proposition}

\begin{proposition}
    \label{prop:e_refs} エントリ $e_n$ は、$n$ の 2 進数表現において最も右に位置する 1 に対応する層別木のルートノードを含んでおり、それより左に位置するすべての層別木のルートノードを参照している。
\end{proposition}

\textbf{ノードの接続}: ここで、$T_n$ に含まれる任意の非葉ノード $b_{i,j}, j > 0$ に対して、$i_\ell,j_\ell$ を左の子ノードの添字、$i_r,j_r$ を右の子ノードの添字とする。$T_n$ に含まれる任意の非葉ノード $b_{i,j}$ に対して左右の子ノードの添字は式 (\ref{eq:children}) で求めることができる。
\begin{equation}
  \left\{
  \begin{aligned}
  i_\ell & = \left\lfloor \frac{i-1}{2^j} \right\rfloor \cdot 2^j+2^{j-1} = (i-1)-((i-1)\bmod  2^j)+2^{j-1} \\
  j_\ell & = j-1 \\
  i_r & = i \\
  j_r & = \lceil \log_2 (i-i_\ell)\rceil
  \end{aligned}
  \right.
    \label{eq:children}
\end{equation}
任意の非葉ノード $b_{i,j}$ とその右の子ノードは同じエントリ $e_i$ に属することから、任意のノードの右の子ノードの $i$ 成分は常に $b_{i,j}$ と同じ値である。

\begin{figure}[htbp]
  \centering
  \includegraphics[scale=.6]{figures/t22-structure-sketch.pdf}
  \caption{$T_{22}$ の木構造: 2 進数表現と層別木 (独立した完全二分部分木) の対応を視覚的に表している。$22$ の 2 進数表現 $\mathtt{10110_2}$ から、$T_{22}$ には $\mathcal{B}'_{22}=\{b'_{16,4},b'_{20,2},b'_{22,1}\}$ をルートとする 3 つの独立した層別木が含まれており、それらを連結する一過性ノード $\{b_{22,5},b_{22,3}\}$ によって単一の木構造を成している。$T_{22}$ のルートから $b_{22}$ に至る経路上のノードはエントリ $e_{22}$ を構成する。}
  \label{fig:t22}
\end{figure}

\textbf{直列化形式}: エントリの時系列順の列 $S_n=(e_1,e_2,\ldots,e_n)$ は木構造 $T_n$ を直列化したログ構造である。
図 \ref{fig:t6_to_t7} は $T_6$ から $T_7$ への木構造の遷移と、直列化形式の変化を示している。

データ列と直列化された木構造の空間計算量は $O(n \log n)$ である。

\textbf{例 2}: $T_n$ に含まれる完全二分部分木の集合を $\mathcal{T}'_n = \{T'_{i,j}\}$ 
\begin{figure}[htbp]
  \centering
  \includegraphics[scale=.8]{figures/t6-to-t7.pdf}
  \caption{$T_{6}$ から $T_7$ への木構造の遷移と直列化形式。}
  \label{fig:t6_to_t7}
\end{figure}

\subsubsection{被覆範囲}

任意の部分木 $T_{i,j} \subseteq T_n$ の被覆範囲について考える。Slate の木の構築方法からその最右葉ノードは $b_i$ である。高さ $j$ の部分木 $T_{i,j}$ の左には、同じ高さ $j$ を持つ別の完全二分部分木が $\lfloor\frac{i-1}{2^j}\rfloor$ 個存在する。高さ $j$ の完全二分木は $2^j$ 個の葉ノードを含んでいることから、最左葉ノードのインデックス $\mathrm{i\_min}(i,j)$ は、$T_{i,j}$ より左に存在する葉ノードの総数 $+1$ であり、したがって式 (\ref{eq:i_min}) となる。
\begin{equation}
    \mathrm{i\_min}(i,j) = \left\lfloor \frac{i-1}{2^j} \right\rfloor \times 2^j + 1 = i - ((i - 1) \bmod 2^j)
    \label{eq:i_min}
\end{equation}

\subsection{Append Operation}

追記 (append) は、既存の木構造 $T_{n-1}$ にデータ $d_n$ を追加して $T_n$ を生成する操作である。これは新しいエントリ $e_n$ を生成し、既存の直列化形式に連結する操作と等価である。このセクションでは、追記のための操作手順を示し、$e_n$ を構築するためには、直前のエントリ $e_{n-1}$ に含まれるノード、またはそのノードが参照しているノードで十分であることを示す。

データ追加に伴って発生するノードは次のように構築する。
\begin{enumerate}
    \item $d_n$ に対応する葉ノード $b_n = \{h_0(d_n),d_n\}$ を生成し、$T_{n-1}$ に含まれる層別木の並びの右に配置する。例えば $n=7$ の場合、$T_6$ に含まれる層別木のルートノードの並びに $b_7$ を連結した $\Sigma_7=(b_{4,2},b_{6,1},b_7)$ を作成する。
    \item  木の並び $\Sigma_n$ を右から 2 個のペアごとに中間ノードで接続するように畳み込む。左右のノードを連結する新しいノードを生成する演算子を $\otimes$ としたとき、$\mathrm{reduce\_right}(\otimes,\Sigma_n)$ を実行する。$b_{i_l,j_l} \oplus b_{i_r,j_r} = b_{i_r,j_l+1}=\{h_{j_l+1}(b_{i_l,j_l}.hash\,||\,b_{i_r,j_r}.hash)\}$
\end{enumerate}

前述の通り、$n$ の 2 進数符号は $T_n$ に含まれる層別木を表していることから、追記による木構造の成長は $n-1$ から $n$ への符号の変化で考えることができる。特に、木構造の成長により複数の層別木がマージされより大きな層別木となることを二進累積すると言う。

\begin{proposition} \label{prop:append}
任意の $n \ge 1$ について、エントリ $e_n$ は直前のエントリ $e_{n-1}$ によって保持される状態と新規データ $d_n$ のみから完全に計算可能である。
\end{proposition}
\begin{proof}
$n=1$ の基底ケースでは $e_1=\{b_1\}$ となることから $d_1$ のみから $e_1$ を計算可能であることは明らかである。

$n-1 \to n$ の帰納ステップについて考える。$n-1$ の 2 進数表現において、最下位ビットから連続する 1 の個数 (つまり最初に 0 が現われる位置) を $\mathrm{cto}(n-1)$ とする。追記操作では、最下位ビットから連続する各 1 に対応する層別木が新しい葉 $b_n$ とマージされ、高さ $\mathrm{cto}(n-1)$ の新しい層別木が形成される (二進累積)。命題 \ref{prop:e_refs} より、マージされる各層別木のルートノードは $e_{n-1}$ に含まれるノードのいずれかと直接接続していることから、この新しい層別木を構築するために必要なノードは $e_{n-1}$ から知ることができる。

次に、$\mathrm{cto}(n-1)$ より大きい桁ではインクリメントの影響を受けず、$n-1$ と $n$ とで同一の値であることから、$e_n$ は $e_{n-1}$ が参照している層別木のルートノードをそのまま引き継ぐことができる。したがって、$e_n$ を構築するためには $e_{n-1}$ と $d_n$ があれば十分である。
\end{proof}

命題 \ref{prop:append} は、Slate で効率的な追記を行うためには、高速なメモリ上に現在のエントリ $e_n$ のみを保持していれば十分であることを示している。以降、本論文では、特に指定しない限り、木構造 $T_n$ において最新のエントリ $e_n$ が高速なメモリ上に保持されていると想定する。

\textbf{追記操作の複雑度}: 木構造 $T_{n-1}$ に新しいデータ $d_n$ を追記する操作では、エントリ $e_n$ を構築するのに要する時間複雑性は $O(\log n)$、エントリ $e_n$ 自身の空間複雑性は $O(\log n)$、エントリ $e_n$ を直列化形式に永続化する I/O 命令複雑性は $O(1)$ である。

\subsection{Get Operation}

本セクションでは Slate における参照操作を形式的に定義する。$\mathrm{get}$ 操作は、木構造 $T_n$ 中の任意のインデックス $i \in [1,n]$ に対応する葉 $b_i$ およびデータ要素 $d_i$ を参照し、また直列化形式の文脈ではその記録位置を特定する。
\begin{equation}
    \mathrm{get}(T_n,i) \to d_i
\end{equation}

\subsubsection{木構造における論理的経路}

\begin{lemma}[層別木の被覆範囲]\label{prop:coverage}
木構造 $T_n$ に含まれる任意の層別木 $T'_{i,j} \subseteq T_n$ の被覆範囲は式 (\ref{eq:coverage}) で表される。
\begin{equation}
    L(T'_{i,j}) = [i - 2^j + 1,i] =[ \left\lfloor \frac{i-1}{2^j} \right\rfloor\times 2^j + 1, i]
    \label{eq:coverage}
\end{equation} 
\end{lemma}
\begin{proof}
式 (\ref{eq:pbst_roots}) より、$T_n$ に高さ $j$ の層別木が存在するための必要条件は $\lfloor n/2^j\rfloor\bmod 2=1$ である。このとき、層別木のルートノード $b'_{i,j}$ は $i=\lfloor n/2^j\rfloor\times 2^j$ である。高さ $j$ の完全二分木は $2^j$ の葉ノードを含み、最右葉は $b_i$ であるため、最左葉は $b_{i-2^j+1}$ である。したがって被覆範囲は式 (\ref{eq:coverage}) となる。
\end{proof}

\begin{lemma}[XOR による層別木の特定]
木構造 $T_n$ において、任意の葉ノード $b_i$ $(1 \le i \le n)$ を含む層別木の高さ $j$ は式 (\ref{eq:i_cover_pbst}) で表すことができる。
\begin{equation}
    j = \left\lfloor \log_2(n \oplus (i-1))\right\rfloor
    \label{eq:i_cover_pbst}
\end{equation}
ここで $\oplus$ はビット単位の排他的論理和 (XOR) を表す。 \label{lemma:ident_pbst}
\end{lemma}
\begin{proof}
式 (\ref{eq:n}) の通り、$n$ と $i-1$ の 2 進数表現をそれぞれ $n=\sum_{k\ge 0}\alpha_k 2^k$、$i-1=\sum_{k\ge 0}\beta_k 2^k$ 、$\alpha_k,\beta_k\in\{0,1\}$ とする。$0 \le i-1 < n$ より、$n$ と $i-1$ の 2 進数表現を上位桁から比較すると、ある位置 $k=p$ で初めて $\alpha_p=1$ かつ $\beta_p=0$ となる。この位置 $p$ より上位のすべての桁 $k \gt p$ では $\alpha_k = \beta_k$ が成り立つ。

XOR 演算の定義より、$n \oplus (i-1)$ の桁 $k$ は $\alpha_k\ne \beta_k$ のとき 1、そうでないとき 0 となる。位置 $p$ は $\alpha_k\ne\beta_k$ を満たす最大の桁 $k$ であるため、$n \oplus (i-1)$ で最も上位の 1 は $p$ に位置する。したがって $p=\lfloor\log_2(n\oplus(i-1))\rfloor$ である。

位置 $p$ において $\alpha_p=1$ であるため、命題 \ref{prop:bit_psbt} より、$T_n$ には高さ $p$ の層別木 $T'_{i^\dagger,p}$ が存在する。補題 \ref{prop:coverage} より、この層別木の被覆範囲は $[i^\dagger - 2^p + 1, i^\dagger]$ である。$k \gt p$ となる桁では $\alpha_k=\beta_k$ であるため
\begin{equation*}
    i - 1 = \sum_{k \ge 0} \beta_k 2^k < \sum_{k \gt p} \alpha_k 2^k + 2^p = i^\dagger
\end{equation*}
また、$\beta_p=0$ かつ $\beta_k \in \{0,1\}$ より
\begin{equation*}
    i-1 = \sum_{k< p} \beta_k 2^k + \sum_{k\gt p} \beta_k 2^k \ge \sum_{k\gt p} \alpha_k 2^k = i^\dagger - 2^p
\end{equation*}
したがって $i^\dagger-2^p \le i-1 < i^\dagger$ であり、$i^\dagger-2^p+1 \le i < i^\dagger$ となる。これは葉 $b_i$ が高さ $p$ の層別木の被覆範囲に含まれていることを示す。
\end{proof}

\begin{lemma}[層別木内のローカルインデックス]
木構造 $T_n$ において、葉 $b_i$ を含む層別木 $T'_{i^\dagger,j}$ とする。層別木 $T'_{i^\dagger,j}$ 内での $b_i$ の 0-indexed なローカルインデックス $k$ は、式 (\ref{eq:i_local_index}) に示すように $i-1$ の下位 $j$ 桁で与えられる。
\begin{equation}
    k = (i-1) \land (2^j-1)
    \label{eq:i_local_index}
\end{equation}
ここで $\land$ はビット単位の論理積を表す。 \label{lemma:i_local_index}
\end{lemma}
\begin{proof}
層別木 $T'_{i^\dagger,j}$ の被覆範囲は $[i^\dagger-2^j+1,i^\dagger]$ であるため、葉 $b_i$ のローカルインデックスは:
\begin{equation*}
    k = i - (i^\dagger-2^j+1) = i-1-(i^\dagger-2^j)
\end{equation*}
補題 \ref{lemma:ident_pbst} の証明より、$n$ と $i-1$ は、2 進数表現において桁 $j$ より上位のすべての桁が一致する。ここで $T'_{i^\dagger,j}$ の左隣の層別木の最右葉 $b_{i^\dagger-2^j}$ の位置は: 
\begin{equation*}
i^\dagger-2^j = \sum_{m\gt j} \alpha_m 2^m = \sum_{m\gt j} \beta_m 2^m = (i-1) \land \lnot(2^{j+1}-1)
\end{equation*}
で表される。これは $i-1$ の 2 進数表現の下位 $j$ 桁を $0$ に設定した値と等しい。これより:
\begin{equation*}
    k = (i-1) - ((i-1) \land \lnot (2^{j+1}-1)) = (i-1) \land (2^j-1)
\end{equation*}
したがって $k$ は $i-1$ の下位 $j$ 桁と等しい。
\end{proof}

\begin{proposition}[完全二分木における経路決定]
高さ $h$ の一般的な完全二分木 $T'$ において、葉ノードを左から順に $0,1,2,\ldots,2^h-1$ とインデックス付けする。ルートノードから葉 $k$ $(0 \le k < 2^h)$ への経路は、$k$ の $h$ 桁 2 進数表現:
\begin{equation*}
    k = \sum_{m=0}^{h-1} k_m 2^m, \quad k_m \in \{0,1\}
\end{equation*}
において、上位ビットから順に、ビット $k_m=0$ なら左枝子ノードへ、$k_m=1$ なら右枝子ノードへ移動することで到達できる。形式的には、高さ $h'$ $(0 < h' \le h)$ のノードから $h'-1$ のノードの遷移方向は:
\begin{equation*}
    \mathrm{direction}(k,h') = \begin{cases}
        \text{left} & \text{if } k_{h'-1} = 0 \\
        \text{right} & \text{if } k_{h'-1} = 1
    \end{cases}
\end{equation*}
で決定する。\label{prop:pbst_dir}
\end{proposition}
\begin{proof}
高さ $h$ に関する帰納法で証明する。

基底ケース: $h=1$ の完全二分木は 2 つの葉を持ち、左の葉がインデックス $0$、右の葉がインデックス $1$ である。従って $k=0$ のとき左枝子ノード、$k=1$ のとき右枝子ノードを正しく選択する。

帰納ステップ: 高さ $h-1$ 以下の完全二分木で命題が成立すると仮定し、高さ $h$ の場合を示す。高さ $h$ の完全二分木は、ルートの左枝と右枝にそれぞれ高さ $h-1$ の完全二分部分木を持つ。葉のインデックス付けに対し、左枝は $[0,2^{h-1}-1]$ を被覆し、右枝は $[2^{h-1},2^h-1]$ を被覆する。

葉 $k$ について、$k < 2^{h-1}$ のとき、$k$ の $h-1$ 桁目は $0$ であり、葉 $k$ は左枝の部分木に含まれているため左枝に移動する。$k \ge 2^{h-1}$ のとき $k$ の $h-1$ 桁目は $1$ であり、葉 $k$ は右枝の部分木に含まれるため右枝に移動する。補題 \ref{lemma:i_local_index} に従い、どちらの部分木へ移動したとしても、$k$ の下位 $h-1$ 桁はその部分木のローカルインデックスを表しており、帰納法により葉 $k$ に到達する。したがって、$k$ の最上位の $h-1$ 桁目が最初の遷移方向を決定し、残りの桁が部分木内での経路を決定する。
\end{proof}

\begin{theorem}[2 進数表現による統一的経路決定]\label{theorem:path}
木構造 $T_n$ において、任意の葉 $b_i$ $(1 \le i \le n)$ への経路は $n$ と $i$ の 2 進数表現のみから以下の手順で決定できる。
\begin{enumerate}
    \item $j = \lfloor\log_2(n \oplus (i-1))\rfloor$ とする。
    \item $T_n$ のルートから $\mathrm{popcount}(\lfloor n / 2^j \rfloor)$ 回、右枝子ノードに移動する。
    \item $j \ne \mathrm{ctz}(n)$ であれば 1 回左枝子ノードに移動する。
    \item $i-1$ の 2 進数表現の下位 $j$ 桁を上位から順に読み、$0$ なら左、$1$ なら右に移動する。
\end{enumerate}
\end{theorem}
\begin{proof}
補題 \ref{prop:bit_psbt} と \ref{lemma:ident_pbst} より、手順 1, 2, 3 は $n$ と $i-1$ の 2 進数表現から葉 $b_i$ を含む層別木を特定し、そのルートノードまで移動する。命題 \ref{prop:pbst_dir} より、手順 4 はその層別木のルートから葉 $b_i$ まで移動する。
\end{proof}

定理 \ref{theorem:path} より、$T_n$ のルートから葉 $b_i$ までの経路長 (ホップ数) は式 (\ref{eq:hops}) で表される。
\begin{equation}
    \mathrm{hops}(n,i) = \mathrm{popcount}(\lfloor n / 2^{j+1} \rfloor) + 1[j \ne \mathrm{ctz}(n)] + j
    \label{eq:hops}
\end{equation}
ここで $j = \lfloor\log_2(n\oplus(i-1))\rfloor$ であり、$1[\cdot]$ は指示関数とする。$\mathrm{popcount}$ と $j$ はそれぞれ $O(\log n)$ であることから、$T_n$ のルートから葉 $b_i$ まで到達する時間複雑性は $O(\log n)$ である。

\textbf{例}: $T_{22}$ から葉 $b_{18}$ への経路について考える。定理 \ref{theorem:path} より:
\begin{eqnarray*}
    n & = & 22 = 10110_2 \\
    i & = & 18 = 10010_2 \\
    n \oplus (i-1) & = & 00111_2 \\
    j & = & \lfloor \log_2(n \oplus (i-1)) \rfloor = 2 \\
    \lfloor n/2^{j+1} \rfloor & = & 22 \ \mathtt{>>} \ 3 = 00010_2 = 2 \\
    \mathrm{popcount}(\lfloor n/2^{j+1} \rfloor) & = & \mathrm{popcount}(2) = 1 \quad \text{(move right branch once)}\\
    \mathrm{ctz}(n) & = & \mathrm{ctz}(22) = 1 \ne j \quad \text{(move left branch)} \\
    k & = & (i - 1) \land (2^j - 1) = 10001_2 \land 00011_2 = 01_2
\end{eqnarray*}
したがって経路は:
\begin{equation*}
    b_{22,5} \overset{\text{right}}{\longrightarrow}
    b_{22,3} \overset{\text{left}}{\longrightarrow}
    b'_{20,2} \overset{0:\text{left}}{\longrightarrow}
    b'_{18,1} \overset{1:\text{right}}{\longrightarrow}
    b_{18} 
\end{equation*}
であり、ホップ数は $\mathrm{hops}(22,18) = 4$ となる。

\subsubsection{読み込みの I/O コスト}
\label{sec:io_cost}

Slate の直列化形式 $S_n=(e_1,e_2,\ldots,e_n)$ において、各エントリ $e_i$ を I/O (seek+read) の単位とする。ここで最新のエントリ $e_n$ はメモリ上に保持されていると仮定する。式 (\ref{eq:eij}) より、エントリ $e_i$ に含まれるノードは、木 $T_i$ のルートから葉 $b_i$ への経路上のすべてのノードと一致する。この経路は、一過性ノードの連鎖と最右の層別木の右辺を経由し、常に右枝方向へ降下する。つまり右枝方向の移動は単一のエントリ内での移動に閉じており、I/O が発生することはない。

木構造 $T_n$ のルートから葉 $b_i$ への経路において、エントリ $e_n$ がメモリ上に保持されているとき、I/O が発生するケースは左枝子ノードへの移動時のみである。したがって葉 $b_i$ を読み取る時に発生する I/O 回数は、経路上の左枝移動回数に等しい\footnote{この経路決定手法は Knuth \cite{knuth1997} EXERCISES 2.3.1-5 で紹介されている、完全二分木のノードインデックス付け手法の応用として導出できる。}。つまり、この構成においての左枝移動回数はキャッシュミス回数に対応する。

\begin{theorem}[I/O 回数]\label{theorem:io}
木構造 $T_n$ において、エントリ $e_n$ から葉 $b_i$ $(1 \le i \le n)$ を参照するための I/O 回数 (キャッシュミス回数) $\mathrm{IO}(n,i)$ は式 (\ref{eq:io}) で与えられる。
\begin{equation}
    \mathrm{IO}(n,i) = 1[j \ne \mathrm{ctz}(n)] + j - \mathrm{popcount}((i-1)\land(2^j-1))
    \label{eq:io}
\end{equation}
ここで $j=\lfloor\log_2(n\oplus(i-1))\rfloor$ である。
\end{theorem}
\begin{proof}
定理 \ref{theorem:path} より、$T_n$ のルートから葉 $b_i$ への経路は、1) ルートから $\mathrm{popcount}(\lfloor n/2^{j+1}\rfloor)$ 回の右枝移動、2) $j\ne\mathrm{ctz}(n)$ であれば 1 回の左枝移動、3) 層別木内で $k=(i-1)\land(2^j-1)$ の 2 進数表現の各桁ごとの左右枝への移動、の 3 フェーズで構成される。フェーズ 1 はすべて右枝移動であるため I/O は発生しない。フェーズ 2 は 0 回か 1 回の左枝移動である。フェーズ 3 では、左枝移動の回数は $k$ の $j$ 桁 2 進数表現における $0$ (左枝移動) の個数と一致する。したがって $j-\mathrm{popcount}(k)=\mathrm{popcount}(\lnot k)$ 回の I/O が発生する。ここで $\lnot k$ は $k$ のビット単位の反転である。
\end{proof}

\begin{proposition}[完全二分木における距離の分布]
高さ $j$ の完全二分木において、ルートから各葉ノードへの I/O 回数 (左枝移動回数) の分布は、パラメータ $(j,0.5)$ の二項分布に従う。
\end{proposition}
\begin{proof}
完全二分木の各葉ノードへの経路は $j$ 回の独立な左右選択で構成される。各選択で左枝を選ぶ確率は $0.5$ であり、左枝移動回数 $k$ となる葉ノードの個数は二項係数 $C(j,k)$ に従う。
\end{proof}

式 (\ref{eq:io}) の第二項 $j−\mathrm{popcount}((i−1)∧(2^j−1))$ は、層別木内でのローカルインデックス $k$ と最右葉 $2^j-1$ との\textbf{ハミング距離} $d_H(k,2^j-1)$ に等しい。これは完全二分木においてルートから葉への経路上の左移動回数が $0$ ビットの個数に対応することによる。 

図 \ref{fig:t352} は $T_{352}$ に対する式 (\ref{eq:io}) の $\mathrm{IO}(352,i)$ の離散分布である。図 \ref{fig:t352_sketch} に概説するように、この分布は $T_{352}$ を構成する 3 つの層別木のハミング距離の分布で構成されている。
\begin{figure}
    \centering
    \includegraphics[width=0.6\linewidth]{figures/io-cost-352.pdf}
    \caption{$T_{352}$ における I/O 回数分布と包絡線。I/O 特性は $T'^{(1)}$, $T'^{(2)}$, $T'^{(3)}$ の 3 つの層別木に対応してセグメント化されており、上限 $c^+(n,i)$ と下限 $c^-(n,i)$ で囲まれた漏斗状の領域が $i=n$ に向かって収束し、Slate の時間局所性を示す。}
    \label{fig:t352}
\end{figure}
\begin{figure}
    \centering
    \includegraphics[width=.8\linewidth]{figures/t352-sketch.pdf}
    \caption{$T_{352}$ の構造と対応するセグメント化された I/O 特性。左図: 木構造 $T_{352}$ は 2 進数表現 $352=2^8 + 2^6 + 2^5$ に対応する高さ $8$, $6$, $5$ の 3 つの層別木 $T'^{(1)}$, $T'^{(2)}$, $T'^{(3)}$ で構成される。一過性ノード (点線の円) が各層別木を接続し、辺のラベルは到達コスト (左枝 $+1$、右枝 $+0$)を示す。 右図: 各層別木における I/O 回数の包絡線。実線は上側包絡線 $c^+$ (紺) と下側包絡線 $c^-$ (緑) を表し、$+1$ の矢印は一過性ノードから層別木への到達時に加算される到達コストを表す。}
    \label{fig:t352_sketch}
\end{figure}

高さ $j$ の一般的な完全二分木において、最左葉のインデックスを $0$、最右葉のインデックスを $2^j-1$ としたとき、任意のインデックス $i$ に対応する葉に到達するための右移動回数は $\mathrm{popcount}(i)$ であり、右移動回数は $\mathrm{popcount}(i \oplus (2^j-1))=j-\mathrm{popcount}(i)$ である。これらはそれぞれ $0$ と $2^j-1$ との\textbf{ハミング距離}の分布と同等となる。図 \ref{fig:pbst_io11} は完全二分木 $T'_{2^{11}}$ (つまり木全体が単一の層別木) における $\mathrm{IO}(2^{11},i)$ の離散分布を表している。
\begin{figure}
    \centering
    \includegraphics[width=0.75\linewidth]{figures/hamming-distance-11.pdf}
    \caption{変数 $x$ の $2^{11}-1$ に対するハミング距離の推移と、その振る舞いを拘束する階段状の境界線。上下の包絡線は、各区間における距離の理論的な最大値および最小値を規定している。}
    \label{fig:pbst_io11}
\end{figure}
この特性により、層別木内はハミング距離で表すことができる。

\subsubsection{I/O コストの上限と下限}

ここで、式 (\ref{eq:io}) に関連する特性として、直列化形式における I/O 回数の上限と下限を導出する。最初に $\mathrm{IO}(n,i)$ の離散分布における上限と下限を定義し、次に完全二分木での上限と下限を定義し、最後に一般的な木構造を分解して最終的な上限と下限を定義する。

式 (\ref{eq:io}) における $\mathrm{IO}(n,i)$ の離散分布の境界をなぞる包絡線 (envelope, 凸包) の上側と下側をそれぞれ $\mathrm{IO}(n,i)$ の $i$ における上限 $c^+(n,i)$ および下限 $c^-(n,i)$ とする。これは図 \ref{fig:pbst_io11} における赤線と青線に相当する。
\begin{figure}
    \centering
    \includegraphics[width=1\linewidth]{figures/t352-sketch.pdf}
    \caption{$T_{352}$ に含まれる層別木の構成とセグメント化された I/O 特性の概念図。}
    \label{fig:t352-sketch}
\end{figure}

まず、完全二分木のケースについて考える。完全二分木 $T'_{n}$ ($n=2^j$) における $\mathrm{IO}(n,i)$ の上限と下限をそれぞれ $c'^+(n,i)$ と $c'^-(n,i)$ とすると、それぞれは式 (\ref{eq:pbst_io_cu})、式 (\ref{eq:pbst_io_cl}) の単調減少関数で表すことができる。
\begin{eqnarray}
c'^+(n,i) & = & \lfloor\log_2(n-i+1)\rfloor \label{eq:pbst_io_cu} \\
c'^-(n,i) & = & \left\lceil\log_2 \frac{n}{i}\right\rceil = j - \lfloor\log_2 i\rfloor \label{eq:pbst_io_cl}
\end{eqnarray}
次に、一般的な木構造 $T_n$ について考える。$T_n$ には $t=\mathrm{popcount}(n)$ 個の層別木が含まれている。その $s \in [1,t]$ 番目の層別木 $T'^{(s)}_{i_s,j_s}$ は葉 $b_i$ を含んでいると想定する。もし $T'^{(s)}_{i_s,j_s}$ が最も右に位置する層別木、つまり $s=t$ であれば、到達コストの項 $1[s<t]$ は $0$ であり、その上限は $c^+=c'^+$ である。このとき、上限 $c^+$ は層別木 $T'^{(s)}$ 内での $i$ の上限 $1[s<t]+c'^+$ と、右の層別木 $T'^{(s+1)}$ の最大値 (つまり高さ) の大きい方である。
\begin{equation}
\begin{aligned}
    c^+(n,i) & = \begin{cases}
        \max\left( 1 + c'^+(2^{j_s},\ell), 1[s+1 < t] + j_{s+1} \right) & \text{if } s < t \\
         c'^+(2^{j_s},\ell) & \text{if } s = t
    \end{cases} \\
    c^-(n,i) & = \begin{cases}
        \min\left( 1[s < t] + c'^-(2^{j_s},\ell), 1\right) & \text{if } s > 1 \\
        1[s < t] + c'^-(2^{j_s},\ell) & \text{if } s = 1
    \end{cases}
\end{aligned}
\label{eq:io_bounds}
\end{equation}

ここで $\ell$ を層別木 $T'^{(s)}_{i_s,j_s}$ 内での $i$ のローカルインデックスとする。式 (\ref{eq:io_bounds}) の最大値と最小値はぞれぞれ式 (\ref{eq:pbst_io_bounds_minmax}) となる。

\begin{equation}
\begin{aligned}
    c^+(n,1) & = c^-(n,1) = \lceil\log_2 n\rceil \\
    c^+(n,n) & = c^-(n,n) = 0
\end{aligned}
\label{eq:pbst_io_bounds_minmax}
\end{equation}

\subsection{Truncate Operation} \label{sec:truncate_operation}

ある世代 $n$ から、それより小さな世代 $m < n$ に戻すには、直列化形式  $S_n=(e_1,e_2,\ldots,e_m,\ldots,e_n)$ を切り詰めて  $S_m=(e_1,e_2,\ldots,e_m)$ にすることができる。

\subsection{Difference Detection Protocol} \label{sec:diff_detect_protocol}

Merkle Tree における包含証明 (inclusion proof) \cite{merkle1989} (または所属証明、認証パス) とは、特定のデータ $d_i$ がツリーに含まれていることを、ツリー全体を開示することなく暗号論的に検証可能にする小さなデータ構造である。この包含証明はルートから葉までの経路上の兄弟ノード (sibling node) のハッシュ値の順序付き列である。Slate における差異検出でも包含証明と類似したハッシュ値の列を用いるが、その構造はやや異なっており、また目的も包含の証明ではない。本論文では、Slate の差異検出のために木を縦方向に割って断面の各ハッシュ値 (層) をサンプリングした列を\textbf{層サンプル} (stratum sample) を導入する。

\subsubsection{Merkle Hash}

一般的な Merkle Tree では、任意のノード $b_{i,j}$ が持つハッシュ値 $c_{i,j}$ は式 (\ref{eq:c}) を満たす。
\begin{equation}
    c_{i,j} = \begin{cases}
        h_j( c_{i_l,j_l} \,||\, c_{i_r,j_r}) & \text{if } j > 0 \\
        h_0(d_i) & \text{otherwise}
    \end{cases}
    \label{eq:c}
\end{equation}
ここで $c_{i_l,j_l}$ と $c_{i_r,j_r}$ はノード $b_{i,j}$ の左右の子ノードのハッシュ値を表している。Merkle Tree の性質より、ハッシュ値 $c_{i,j}$ は対応するノード $b_{i,j}$ をルートとする部分木に含まれるすべてのデータの暗号論的ダイジェストを意味する。したがって、2 つの木構造 $T^A_n$ と $T^B_n$ のルートハッシュ $c^A_n, c^B_n$ が異なる場合、暗号論的に $T^A_n$ と $T^B_n$ には異なるデータが少なくとも 1 つ含まれていることを意味している。

\subsubsection{Stratum Sampling}

Slate の木構造 $T_n$ に含まれる任意の部分木 $T_{i,j} \subseteq T_n$ のルートノード $b_{i,j}$ から、部分木の最右葉ノード $b_i$ への経路を式 (\ref{eq:path}) のように定義する。
\begin{equation}
    \mathcal{P}_{i,j} = \{ b_{i,k} \mid 0 \le k \le j \}
    \label{eq:path}
\end{equation}
この経路は、高さ $j$ のルートノードから 0 の葉まで、常に右枝に向かって降下する。式 (\ref{eq:children}) より、$\mathcal{P}_{i,j}$ 上のノード $b_{i,k} (k > 0)$ の左子ノードを式 (\ref{eq:left_child}) のように定義する。
\begin{equation}
    \mathrm{leftchild}(b_{i,k}) = \begin{cases}
        b_{i-2^{k-1},k-1} & \text{if } k > 0 \land i \ge 2^{k-1} \\
        \bot & \text{otherwise}
    \end{cases}
    \label{eq:left_child}
\end{equation}
ここで $\bot$ は左子ノードが存在しないことを示す。

\begin{definition}[層サンプル]
木構造 $T_n$ に含まれる任意の部分木 $T_{i,j} \subseteq T_n$ において、$T_{i,j}$ のルートノード $b_{i,j}$ から部分木の最右葉ノード $b_i$ に至る経路 $\mathcal{P}_{i,j}$ 上の各ノードの左子ノードのハッシュ値と、最右葉 $b_i$ のハッシュ値の集合を\textbf{層サンプル} (stratum sample) と定義する。\label{def:stratum_sample}
\begin{equation}
    \mathcal{C}_{i,j} = \{ c_{i',j'} \mid b_{i',j'} = \mathrm{leftchild}(b_{i,k}), b_{i,k} \in \mathcal{P}_{i,j}, b_{i',j'} \ne \bot \} \cup \{ c_{i,0} \}
    \label{eq:ss}
\end{equation}
\end{definition}

層サンプルの各 $c_{i',j'}$ の添字は式 (\ref{eq:children}) で求めることができる。

\begin{lemma}[層サンプルの高さ上限]
$j>0$ である層サンプル $\mathcal{C}_{i,j}$ に含まれる任意のハッシュ値 $c_{i',j'}$ に対応するノード $b_{i',j'}$ の高さは、ルートノード $b_{i,j}$ より必ず小さい。
\begin{equation}
    \forall c_{i',j'} \in \mathcal{C}_{i,j}, j > 0: j' < j
\end{equation} \label{lemma:c_height}
\end{lemma}
\begin{proof}
定義 \ref{def:stratum_sample} より、層サンプル $\mathcal{C}_{i,j}$ のすべてのハッシュ値は以下のいずれかのノードに由来する:
\begin{enumerate}
    \item ルートノード $b_{i,j}$ から最右葉 $b_i$ への経路上の各ノードの左子ノード
    \item 最右葉 $b_i$ 自身 (高さ $j=0$ のノード)
\end{enumerate}
$j > 0$ より 2. のケースは除外できる。ルートノード $b_{i,j}$ は $\mathcal{P}_{i,j}$ に含まれるノードのなかでもっとも大きい高さ $j$ を持つ。式 (\ref{eq:children}) より、$\mathcal{P}_{i,j}$ 上の各ノードの左子ノードの高さはその親ノードより厳密に 1 つ小さい。したがって、層サンプル $\mathcal{C}_{i,j}$ に含まれるすべてのハッシュ値 $c_{i',j'}$ に対して $j' < j$ が成立する。
\end{proof}

\begin{lemma}[層サンプルの完全被覆性]
木構造 $T_n$ に含まれる任意の部分木 $T_{i,j} \subseteq T_n$ において、その層サンプル $\mathcal{C}_{i,j}$ の各ハッシュ値 $c_{i',j'}$ に対応するノード $b_{i',j'}$ を頂点とする各部分木は、部分木 $T_{i,j}$ の葉ノード全体をカバーする。
    \begin{equation}
        L(T_{i,j}) = \bigcup_{c_{i',j'} \in \mathcal{C}_{i,j}} L(T_{i',j'}) = [i-((i-1)\bmod{2^j}), i]
    \end{equation}
    ここで $L(T_{i,j})$ は部分木 $T_{i,j}$ の被覆範囲 (式 (\ref{eq:i_min}) も参照) を表す。 \label{lemma:cover} 
\end{lemma}
\begin{proof}
$T_{i,j}$ のルート $b_{i,j}$ から葉 $b_i$ までの経路上のノード集合を $\mathcal{P}_{i,j}=(v_k,\ldots,v_0)$ とする。ここで $v_0$ は葉ノード、$v_k$ はルートノードである。各 $v_j$ に対してその兄弟ノードを $s_j$ とする。二分木構造より、各段階で:
\begin{equation*}
    L(v_{j+1}) = L(v_j) \cup L(s_j), \quad
    L(v_j) \cap L(s_j) = \emptyset
\end{equation*}
が成り立つ。これを再帰的に適用すると:
\begin{equation*}
    L(v_k) = L(v_0) \cup \bigcup_{j=1}^{k} L(s_j)
\end{equation*}
を得る。$L(v_k)$ はルートノードの被覆範囲 $L(b_{i,j})$ である。また $L(v_0)=L(b_i)$ である。$ \bigcup_{j=1}^k L(s_j)$ は $\mathcal{P}_{i,j}$ の各左子ノードの被覆範囲を表している。この構造は定義 \ref{def:stratum_sample} の層サンプル $\mathcal{C}_{i,j}$ と同じであるため、層サンプル $\mathcal{C}_{i,j}$ は木 $T_{i,j}$ の葉を完全に被覆する。各 $s_j$ は異なる高さで $L(v_j)$ の補集合として現われるため、互いに独立している。

部分木コミットメント列は包含証明に葉ノード $b_i$ のコミットメント (ハッシュ値) を含めた集合であるため、任意の $1 \le i \le n$ に対し、部分木コミットメント列 $C_{n,i}$ は $T_n$ に含まれるすべての葉ノードを被覆する。
\end{proof}

\begin{lemma}[層サンプルの暗号論的ダイジェスト性]
層サンプル $\mathcal{C}_{i,j}$ は、木 $T_{i,j}$ のルートハッシュ $c_{i,j}$ と同様に、その木に含まれるすべての葉に含まれるデータの暗号論的ダイジェストを表す。 \label{lemma:sample_digest}
\end{lemma}
\begin{proof}
補題 \ref{lemma:cover} とハッシュ関数 $h_j$ の衝突困難性より、層サンプルはその木のすべての葉ノードを被覆するため、各部分木のルートハッシュ値の集合は木全体のダイジェストと同等である。
\end{proof}

\begin{lemma}[層サンプルの等価性]
2つの $n$-世代の木構造  $T^A_n$ と $T^B_n$ の部分木 $T^A_{i,j} \subseteq T^A_n$ と $T^B_{i,j} \subseteq T^B_n$ において、双方の層サンプルが等しい場合、暗号論的に $T^A_{i,j}$ と $T^B_{i,j}$ の被覆範囲に含まれる葉のデータはすべて等しい。 \label{lemma:c_identical}
\begin{equation*}
    \mathcal{C}^A_{i,j} = \mathcal{C}^B_{i,j} \quad\Longleftrightarrow \quad \forall k \in [\mathrm{i\_min}(i,j), i] : d^A_k = d^B_k
\end{equation*}
\end{lemma}
\begin{proof}
補題 \ref{lemma:sample_digest} とハッシュ関数 $h_j$ の衝突困難性より、葉のケースでは $\mathcal{C}^A_{i,0}=\mathcal{C}^B_{i,0}$ は $h_0(d^A_i)=h_0(d^B_i)$ を意味しているため $d^A_i=d^B_i$ は明らか。中間ノードのケースも同様に $c^A_{i,j}=c^B_{i,j}$ ならば双方の $T_{i,j}$ の被覆範囲の葉のデータは暗号論的に等しい。補題 \ref{lemma:cover} より、層サンプル $\mathcal{C}_{i,j}$ は $T_{i,j}$ のすべての葉のデータを被覆するため、双方の $T_{i,j}$ に含まれるデータはすべて等しいことが証明される。
\end{proof}

\subsubsection{Difference Detection using the Stratum Sample}

差異位置の検出は 2 者間での層サンプルを 1 回以上交換することによって達成される。$A$ と $B$ の 2 者がそれぞれツリー $T_{n_a}$ と $T_{n_b}$ を保持していると想定する。ここで $A$ と $B$ は $n=\min(n_a,n_b)$ とする $n$-世代のツリー $T_n$ 上で動作することに合意しているものとする (世代が大きい方の木は $n$ 世代のスナップショットを使用する)。

発起者である $A$ は、$B$ の持つデータ列との差異を検出するために、まず最初のラウンドで自身の木構造 $T^A_n$ 全体を対象とする層サンプル $\mathcal{C}^A_{n,\lceil\log_2 n\rceil}$ を作成して $B$ に送信する。$B$ はその各要素を自身の同じ部分木の層サンプル  $\mathcal{C}^B_{n,\lceil\log_2 n\rceil}$  と比較する。双方の層サンプルが同一であれば、補題 \ref{lemma:c_identical} より、$A$ と $B$ の $n$-世代までのデータ列に差異がないことが暗号論的に証明され、$B$ は $\bot$ で応答する。

一方、層サンプルに含まれる 1 つ以上のハッシュ値が異なっていた場合、それらに対応する部分木の範囲に差異が存在することが明らかになる。ここで我々は、データ列の分岐を検出することを目的としているため、最初に出現する差異の位置を特定したいと考えている。したがって、差異のあるハッシュ値 $c^A_{i^*,j^*}\ne c^B_{i^*,j^*}$ の中で $i^*$ が最も小さい部分木に目的の差異が含まれていることになる。

ここで $j^*=0$ であれば葉ノードの差異を意味しており、位置 $i^*$ のデータ $d^A_{i^*}$ と $d^B_{i^*}$ が異なることが明らかとなるため、$B$ は $\mathrm{OK}\{i^*\}$ で応答する。そうでない場合、$B$ は $b_{i^*,j^*}$ をルートとする部分木の層サンプル $\mathcal{C}^B_{i^*,j^*}$ を作成して $A$ に送信し、送信役/受信役を入れ替えて第二ラウンドを開始する。このインタラクションを繰り返すことにより、$A$ と $B$ のどちらかは最終的に目的の差異の位置 $i^*$ を発見する。

差異検出を定式化した関数として表現すると式 (\ref{eq:finddiff}) のようになる。
\begin{equation}
    \mathrm{finddiff}(\mathcal{C}^\mathrm{sender}_{i,j},\mathcal{C}^\mathrm{receiver}_{i,j}) = \begin{cases}
        \bot & \text{if } \mathcal{C}^\mathrm{sender}_{i,j} = \mathcal{C}^\mathrm{receiver}_{i,j} \\
        \tilde{i} & \text{if } \delta_{i,j} = \{(\tilde{i},0)\}\\
        \mathrm{finddiff}(\mathcal{C}^\mathrm{receiver}_{\tilde{i},\tilde{j}},\mathcal{C}^\mathrm{sender}_{\tilde{i},\tilde{j}}) & \text{otherwise}
    \end{cases}
    \label{eq:finddiff}
\end{equation}
ここで $\delta_{i,j}$ は $\mathcal{C}^A_{i,j}$ と $\mathcal{C}^B_{i,j}$ において $c^A_{i^*,j^*}\ne c^B_{i^*,j^*}$ であるような添字 $(i^*,j^*)$ の集合、$\tilde{i},\tilde{j}$ は $\delta_{i,j}$ の中で最も小さい $i^*$ を持つ添字ペアを表す。
\begin{eqnarray*}
    \delta_{i,j} & = & \{ (i^*,j^*) \mid c^A_{i^*,j^*} \in \mathcal{C}^A_{i,j}, c^B_{i^*,j^*} \in \mathcal{C}^B_{i,j}, c^A_{i^*,j^*} \ne c^B_{i^*,j^*} \} \\
    \tilde{i} & = & \min\{ i^* \mid (i^*, j^*) \in \delta_{i,j} \} \\
    \tilde{j} & = & j^*: (i^*,j^*) \in \delta_{i,j}, i^*=\tilde{i}
\end{eqnarray*}
初期実行は $\mathrm{finddiff}(\mathcal{C}^A_n,\mathcal{C}^B_n)$ で開始する。

\begin{theorem}[差異検出の正当性]
$T^A_n$ と $T^B_n$ を世代 $n$ のツリーとし、$D^A[1..n]$ と $D^B[1..n]$ をそれぞれのデータ列とする。ハッシュ関数 h が暗号論的に衝突困難であると仮定する。このとき、$\mathrm{finddiff}(\mathcal{C}^A_n,T^B_n)$ は以下を満たす:
    \begin{enumerate}
        \item $D^A[1..n] = D^B[1..n]$ ならば $\bot$ を返す。
        \item そうでなければ $\tilde{i} = \min \{i \in [1,n] \mid D^A[i] \ne D^B[i]\}$ を返す。
    \end{enumerate}
\end{theorem}
\begin{proof}
補題 \ref{lemma:sample_digest} より、$\mathcal{C}_{n}$ に含まれるすべてのハッシュ値が等しい場合、$T^A_n$ と $T^B_n$ は同じデータ列を持っており、差異は存在しない ($\bot$ となる)。

$\Delta$ を $D^A_n$ と $D^B_n$ の差異の位置の集合とする。つまり $\Delta=\{i\in[1,n]\mid D^A[i]\ne D^B[i]\}$ である。ある部分木 $T_{i,j} \subseteq T_n$ のカバーする範囲に差異が含まれている場合、$T_{i,j}$ のカバー範囲に $\Delta$ の少なくとも一つの要素が含まれている。
\begin{equation*}
    \Delta \cap L(T_{i,j}) \ne \emptyset
\end{equation*}
この場合、層サンプル $\mathcal{C}_{i,j}$ のうち、少なくとも 1 つのハッシュ値が異なっているはずである。補題 \ref{lemma:c_height} より、異なるハッシュ値 $c_{\tilde{i},\tilde{j}}$ に対応する部分木 $T_{\tilde{i},\tilde{j}}$ の高さは元の部分木 $T_{i,j}$ の高さより小さい。つまり、$\tilde{j} < j$ であり $T_{\tilde{i},\tilde{j}} \subset T_{i,j}$ である。

したがって、どのハッシュ値に差異があっても、元の $T_{i,j}$ より小さな範囲をカバーする部分木を選択することになる。したがって、$\mathrm{finddiff}$ のラウンドが進むにつれ被覆範囲は必ず小さくなり、最終的に高さゼロの部分木、つまり差異を含む葉ノード $b_{\tilde{i}}$ を特定する。
\end{proof}

差異の特定は、データ列の中で最も古いもののみならず、すべての差異を特定できると予想される。ただし、そのような手法はデータ列の分岐点を検出することが目的の本論文の範囲外である。

\begin{theorem}[差異検出の終了性] \label{theorem:diffdetect_finiteness}
$\mathrm{finddiff}(\mathcal{C}^A_n,\mathcal{C}^B_n)$ は、$\tilde{i}=n$ のとき最良ケースで 1 回、$\tilde{i}=1$ のとき最悪ケースで $\lceil \log_2 n \rceil$ 回のラウンドで終了する。
\end{theorem}
\begin{proof}
最初のラウンドで交換される層サンプル $\mathcal{C}_{n,\lfloor\log_2 n\rfloor}$ は葉ノード $b_n$ のハッシュ値 $c_{n,0}$ を含んでいるため、$\tilde{i}=n$ のとき 1 回目のラウンドで $\tilde{i}=n$ を特定する。一方、$\mathcal{C}_{n,\lfloor\log_2 n\rfloor}$ は $c_{n,0}$ 以外はすべて中間ノードのハッシュであるため、$c_{n,0}$ 以外で差異を検出した場合は $\tilde{i}$ の特定のために追加のラウンドが必要となる。したがって、最良ケースは $\tilde{i}=n$ のケースで 1 回のラウンドで終了する。

最悪ケースはすべてのラウンドで層サンプル $\mathcal{C}_{i,j}$ の中から最も高い部分木 (最も左の部分木) を選択し続ける。1 回のラウンドで高さが 1 しか減少しないため、葉ノード ($j=0$) に到達するためには木 $T_n$ の高さである $\lceil\log_2 n\rceil$ 回のラウンドが必要である。この過程では左枝方向に降下して行くため、最終的に $\tilde{i}=1$ に到達する。したがって、最悪ケースは $\tilde{i}=1$ のケースで $\lceil\log_2 n\rceil$ 回のラウンドで終了する。
\end{proof}

さらなる改善として、目的とする差異が特定の葉 $b_i$ かその近傍に存在する可能性が高いことが事前に分かっていれば、最初の層サンプルを $\mathcal{C}_i$ から開始することで差異検出の効率を高められると考えられるが、その議論はこの論文では範囲外である。

\subsection{Complexity Analysis}

本セクションでは、前節までに述べた Slate の主要な操作について時間複雑度 (計算複雑度)、空間複雑度、および I/O 複雑度を分析する。現実の大規模なデータ構造では、CPU 演算と比較して二次記憶装置へのアクセス遅延が数桁大きくなり、I/O 回数がシステム全体の性能を支配する要因となる。したがって Slate では従来の計算複雑度に加えて I/O 複雑度を明示的に評価し、Slate が I/O コストの観点からも効率的であることを示す。 

以下の分析では、木構造 $T_n$ に $n$ 個の葉ノードが含まれており、最新のエントリ $e_n$ を一時記憶装置 (メモリ) 上に保持している状況を想定する。

\textbf{Append 操作}. 木構造 $T_{n-1}$ に新しいデータ $d_n$ を追加してエントリ $e_n$ を作成し $T_n$ に移行する操作である。命題 \ref{prop:append} より、$e_n$ を構築するには $e_{n-1}$ と $d_n$ のみが必要である。$e_n$ に含まれるノード数は $|J'(n)| + |J(n)|$ であるため時間複雑度、空間複雑度は $O(\log n)$ であり、ハッシュ計算は各ノードあたり定数時間で完了する。直列化形式への永続化は $e_n$ を既存のログ末尾に追記する単一の書き込み操作で完了する。

\textbf{Get 操作}. インデックス $i$ に対応する葉ノード $b_i$ を参照する操作である。定理 \ref{theorem:path} より、$T_n$ のルートノードから $b_i$ への経路長は $O(\log n)$ である。定理 \ref{theorem:io} より、その経路で $b_i$ に到達するときの I/O 回数 $\mathrm{IO}(n,i)$ は経路上の左枝移動回数に等しく、式 (\ref{eq:io}) で与えられる。

\textbf{Truncate 操作}. 世代 $n$ から $m < n$ への切り詰めは、直列化形式 $S_n$ の末尾 $n-m$ 個のエントリを理論的に無効化する操作である。ファイルシステムの truncate システムコールを使用する場合、その位置の特定に Get 操作と同じ複雑度を要し、物理的なデータ削除は定数時間で完了する。

\textbf{Difference Detection 操作}. 2 つの木 $T^A_n$ と $T^B_n$ の分岐点 (または共通接頭辞) を特定する操作である。定理 \ref{theorem:path} より、必要なラウンド数は最良で 1 回、最悪で $\lceil\log_2 n\rceil$ で、それぞれのラウンドに $O(\log n)$ 回の読み込みと比較が行われる。各ラウンドで交換される層サンプルのサイズは $O(\log n)$ である。

表 \ref{table:complexities} に各操作の複雑度を要約する。
\begin{table}[t]
  \centering
  \begin{tabular}{lllll}
   \hline
    操作 & 時間複雑度 & 空間複雑度 & I/O 複雑度 (Write) & I/O 複雑度 (Read)  \\
    \hline \hline
    Append & $O(\log n)$ & $O(\log n)$ & $O(1)$ & $0$ \\ \hline
    Get (最悪) & $O(\log n)$ & $O(1)$ & ─ & $O(\log n)$ \\ \hline
    Get (最良) & $O(1)$ & $O(1)$ & ─ & $0$ \\ \hline
    Get (期待) & $O(\log(n-i))$ & $O(1$) & ─ & $O(\log(n-i))$ \\ \hline
    Truncate & $O(1)$ & $O(1)$ & $O(1)$ & ─ \\ \hline
    Diff Detection (最悪) & $O(\log^2 n)$ & $O(\log n)$ & ─ & $O(\log^2 n)$ \\ \hline
    Diff Detection (最良) & $O(\log n)$ & $O(\log n)$ & ─ & $O(\log n)$ \\ \hline
   \end{tabular}
\caption{各操作の複雑度}
\label{table:complexities}
\end{table}

\subsubsection{空間複雑度}

木構造 $T_n$ 全体の空間複雑度は、各エントリ $e_n$ に含まれるノード数が $O(\log n)$ であることから、直列化形式 $S_n=(e_1,e_2,\ldots,e_n)$ の総ノード数は式 (\ref{eq:s_complexity}) で与えられる。
\begin{equation}
\sum_{i=1}^n |e_i| = \sum_{i=1}^n O(\log i) = O(n \log n)
\label{eq:s_complexity}
\end{equation}
一方、メモリ上に保持する必要があるのは最新エントリ $e_n$ のみであり、その空間複雑度は $O(\log n)$ である。

\subsubsection{I/O 複雑度の期待値}

式 (\ref{eq:io}) の I/O 回数 $\mathrm{IO}(n,i)$ はアクセス対象のインデックス $i$ に依存する。時間局所性を持つワークロードでは、$i$ が $n$ に近い値に偏るため、I/O 回数の期待値は最悪ケースより大幅に小さくなる。

アクセス確率が幾何分布 $\mathrm{Pr}[\text{access } i] \propto p^{n-i}(1-p)$ に従うと仮定した場合、I/O 回数の期待値 $\mathbb{E}[\mathrm{IO}(n,i)]$ は式 (\ref{eq:expected_io_on_geodist}) で近似できる。
\begin{equation}
\mathbb{E}[\mathrm{IO}(n,i)] \approx \sum_{k=0}^{\lfloor\log_2 n\rfloor} k \cdot \mathrm{Pr}[\mathrm{IO} = k]
\label{eq:expected_io_on_geodist}
\end{equation}
式 (\ref{eq:pbst_io_cu}) および (\ref{eq:pbst_io_cl}) の包絡線により、I/O 回数 $k$ を達成する葉ノードの個数は、最新から距離 $2^{k-1}$ から $2^k-1$ の集中する。したがって、時間局所性が強いワークロード ($p$ が $1$ に近い) では、期待 I/O 回数は $O(1)$ に近づく。

Slate は History Tree と同等の空間複雑度を持ちながら、追記の I/O を $O(1)$ に改善し、さらに最新データへのアクセスを $O(1)$ で実現する。この非対称な性能特性は、追記偏重かつ時間局所性を持つワークロードにおいて顕著な優位性をもたらす。これらの正しさは、基盤となるハッシュ関数 $h$ の衝突困難性に帰着する。

\subsection{並行アクセスモデル} \label{sec:concurrent_access_model}

このセクションでは Slate において Truncate 操作を想定しない、あるいは起きないことが保証されている状況での並行アクセスについて分析する。この場合、Slate は部分永続的データ構造の性質を持ち、追加の同期機構なしで安全な並行アクセスを可能にする。本セクションでは Slate が提供する並行性保証を形式化する。

\subsubsection{Single-Write / Multi-Reader 実行モデル}

Slate は Single-Writer/Multi-Reader (SWMR) 並行実行モデルと適合する。このモデルでは:
\begin{itemize}
    \item 単一の Writer が Append 操作を実行し、新しいエントリ $e_n$ を直列化形式に追記する。
    \item 複数の Reader が同時に Get 操作を実行できる。
\end{itemize}

永続性ノード $b'_{i,j}$ は一度作成されると不変であるため、Reader が参照するノードは Writer によって変更されることはない。

\subsubsection{スナップショット読み取り一貫性}

Reader は木構造 $T_n$ のメモリ上に保持されている最新のエントリ $e_n$ を読み取るか、直列化形式から過去の $e_m$ ($m \le n$) を読み取ることで、読み取り時点で読み取り対象とする世代を特定する。この読み取りに並行して別の Writer が $e_{n+1}$ 以降を追加しても、$e_n$ または $e_m$ を起点とする Reader の参照経路からそれらが観測されることはない。したがって Slate の読み取りはスナップショットに基づく一貫した結果が保証される。

\subsubsection{Trancate の考察}

Truncate 操作は世代を減少させることから、並行読み取りと競合する可能性がある。実用的な実装で Truncate をサポートするためにはいくつかの選択肢がある。例えばコミット (確定) 済みの世代を保護領域として Truncate 操作が及ばないようにする、Copy-on-Write を実装する、MVCC のような世代の分岐をサポートする、といった方法が考えられる。

\section{Experimental Evaluation}

このセクションでは Rust 実装を用いた Slate の実験的評価を説明する。評価では、追記性能、参照性能、時間局所性を持つワークロードでの参照性能、および差異検出性能について、複数のデータ構造や組み込みデータベースと比較する。

\subsection{Implementation}

Slate の参照実装は Rust で記述されており、\href{https://github.com/torao/slate}{https://github.com/torao/slate} から参照可能である。実装は以下の設計方針に基づく:

\begin{enumerate}
    \item \textbf{直列化形式}: エントリ単位でのシーケンシャル追記を基本とし、各エントリは固定長ヘッダと可変長のノード列で構成される。なお直列化形式に必要なストレージは「次に書き込むべき位置」をサポートすれば良く、比較のために RocksDB 実装を追加している。
    \item \textbf{ハッシュ関数}: BLAKE3 をノードの高さ $j$ ごとに $h_j(m) = H(j\,||\,m)$ とドメイン分離して計算する。
    \item \textbf{キャッシュ}: 最新エントリ $e_n$ をメモリ上に保持し、オプションで複数エントリのキャッシュをサポートする。
\end{enumerate}

\subsection{Experimental Setup}

本実験は Amazon Web Services (AWS) のクラウド環境上で実施した。実験には計算最適化インスタンス (c5.2xlarge) であり、詳細はハードウェアおよびソフトウェア構成は表 \ref{table:experimental_setup} の通りである。ベンチマークプログラム\footnote{ベンチマークプログラムは \href{https://github.com/torao/slate-benchmark}{https://github.com/torao/slate-benchmark} で利用可能。}は Rust (rustc 1.90) でコンパイルされ、\verb|--release| フラグを用いて最適化されている。 

\begin{table}
    \centering
    \begin{tabular}{ll}
        \hline
         Component & Specification \\
         \hline \hline
         CPU & Intel(R) Xeon(R) Platinum 8124M @ 3.00GHz (4 Cores, 8 Threads)  \\ \hline
         Memory & 15.1GB \\ \hline
         Storage & NVMe SSD (AWS EBS), ext4 Filesystem  \\ \hline
         Disk Performance & Seq. Write: 149.0 MB/s, Random IOPS: \~1,699  \\ \hline
         OS & Ubuntu 24.04.3 LTS (Kernel 6.14.0-1012-aws)  \\ \hline
         Cloud Provider & AWS EC2 (Region: us-east-1)  \\ \hline
    \end{tabular}
    \caption{実験環境のハードウェア・ソフトウェア構成}
    \label{table:experimental_setup}
\end{table}

\begin{table}
    \centering
    \begin{tabular}{lll}
        \hline
        Structure& Storage & Description \\
        \hline \hline
         Slate (file) & Sequential File  & Slate の追記ストレージ版 \\ \hline
         Slate (rocksdb) & RocksDB & Slate の LSM-Tree ストレージ版 \\ \hline
         Slate (memkvs) & Memory & (参考) 同一構造で非永続な記憶装置 \\ \hline
         Binary Merkle Tree & Random Access File & $2^h$ 個の葉を持つ静的な Merkle Tree \\ \hline
         Unindexed Sequential File & Sequential File & (参考) 単純な固定長データ列ファイル\\ \hline
         IAVL+ & LevelDB & Cosmos で使用されている平衡木\\ \hline
         Prolly Tree & DoltDB & git に類似したデータ操作をサポートする DB\\ \hline
    \end{tabular}
    \caption{実験に使用したデータ構造と実装}
    \label{tab:experimental_data_structures_and_impls}
\end{table}

\subsection{Storage Space Consumption} \label{sec:storage_space_consumption}

Slate の直列化形式における空間消費量を評価する。式 (\ref{eq:s_complexity}) より、Slate の理論的空間計算量は $O(n \log n)$ である。本セクションでは、この理論値と実測値の整合性を確認すると共に、他のデータ構造との比較を行う。

データ数 $n$ を変化させたときの各実装のストレージ占有サイズの計測値を図 \ref{fig:ex_storage_space} に示す。Slate (file) は $n=8\mathrm{Mi}$ において約 $5.4$ GB を消費している。参照実装の直列化形式では、各エントリ $e_i$ は 65 bytes の固定長部分 (インデックスや葉ノードなど) と、各 50 bytes の可変長部分 (中間ノード) で構成される。ここで、式 (\ref{eq:ei_size}) より、$n$ 個のエントリに含まれる中間ノードの総数は $M(n) = \sum_{i=1}^n(\mathrm{ctz}(i)+\mathrm{popcount}(i)-1)$ となる。したがって $n=8\mathrm{Mi}=2^{23}$ においては、直列化形式 $S_{8\mathrm{Mi}}$ は $M(2^{23})=23\times 2^{22}$ 個の中間ノードを含み、そのサイズの理論値は $|S_{8\mathrm{Mi}}|=65\times 8\mathrm{Mi}+50\times M(2^{23})=5\mathrm{GiB}$ となる。したがってこの計測値は理論値と整合している。

Slate (rocksdb) は同一の理論構造ながら Slate (file) より大きいストレージを消費している。これは RocksDB の LSM-Tree 構造に伴うメタデータと、コンパクション処理前の冗長データに起因する。

DoltDB と IAVL+ はデータ作成のための Append 操作に多くの時間を費やしたため、$n=1\mathrm{Mi}$ 程度で計測を打ち切った。DoltDB では特に急速なストレージ消費が観測されている。

これらの結果は、汎用的なバージョン管理データ構造や平衡木ベースの認証付きデータ構造と比較して、Slate が効率的な空間消費を達成していることを示している。
\begin{figure}
    \centering
    \includegraphics[width=0.75\linewidth]{figures/20250920202203-volume.pdf}
    \caption{各実装におけるストレージ消費量の比較。$n$ 個のデータを保存した時点でストレージ上で占有しているデータ領域のサイズを示す。}
    \label{fig:ex_storage_space}
\end{figure}

\subsection{Append Performance} \label{sec:append_performance}

図 \ref{fig:append_performance} に $n$ 個のデータを順次追加するのに要した累積時間を示す。図 \ref{fig:append_performance}(a) は、任意の追加をサポートしない Binary Merkle Tree を除いた比較対象を含み、図 \ref{fig:append_performance}(b) は特性の違いを明確にするため IAVL+ と DoltDB を除外している。

\textbf{全体的傾向}. 図 \ref{fig:append_performance}(a) では、IAVL+ (LevelDB) と DoltDB が他の実装と比べて著しく遅いことが観測される。$n=1,048,576$ の追加において IAVL+ は $318,188$ [msec] を要しており、これは Slate (file) の約 80 倍に相当する。これらのデータストアは汎用的な Key-Value ストアやバージョン管理データベースとして設計されており、追記専用ワークロードに最適化されていないことが主な要因と考えられる。特に IAVL+ は各追記操作で AVL 木の再平衡化と LevelDB への書き込みが発生し、DoltDB は Prolly Tree の確率的分割とインデックス更新のオーバーヘッドを伴う。

\textbf{Slate の各実装の比較}. 図 \ref{fig:append_performance}(b) より、同一の Slate 構造であっても物理ストレージの選択が性能に大きく影響することが確認できる。$n=8\times 10^6$ において、Slate (rocksdb) は約 $1.5\times 10^5$ [msec] を要するのに対し、Slate (file) は約 $4.3\times 10^4$ [msec] で完了している。RocksDB の実装は LSM-Tree のコンパクションに起因するレイテンシによる変動が顕著であり、データ量の増加に伴い性能劣化が加速する傾向を示す。一方、Slate (file) はシーケンシャル追記のみで完結するため、$n$ に対してほぼ線形の増加を維持している。

Unindexed Sequence File (ハッシュツリーを持たない単純な追記専用ログ) は最も高速であり、$n=8\times 10^6$ で約 $1.0 \times 10^4$ [msec] を達成している。Slate (file) はこのベースラインの約 4 倍の時間を要するが、これはエントリごとに $O(\log n)$ 個のノードのハッシュ計算と書き込みが発生することに起因する。この追加コストは、分岐検出や完全性検証と言った機能を提供するための合理的なトレードオフである。

Slate (memkvs) は非永続的なインメモリ実装であり、I/O とノード直列化のオーバヘッドを排除した場合の性能上限を示す。図 \ref{fig:append_performance}(b) において Slate (file) と Slate (memkvs) の差は比較的小さく、Slate の追記操作がシーケンシャル I/O として効率的に実行されていることを示唆している。

これらの結果は Slate の追記最適化設計が有効であることを実証している。特に、エントリ単位でのシーケンシャルファイル追記という直列化形式の選択により、汎用 Key-Value ストア (RocksDB) を使用する場合と比較して 3 倍以上の性能向上を達成した。また、IAVL+ や DoltDB のような既存の認証付きデータ構造と比較して 1-2 桁の性能差があることは、追記偏重ワークロードにおいて専用設計のデータ構造が重要であることを示している。

\begin{figure}[t]
  \centering
  \begin{minipage}{0.32\linewidth}
    \centering
    \includegraphics[width=1\linewidth]{figures/20250920202203-append-whole.pdf}
    \caption*{(a) 対象となる実装全体の比較}
  \end{minipage}\hfill
  \begin{minipage}{0.64\linewidth}
    \centering
     \includegraphics[width=1\linewidth]{figures/20250920202203-append.pdf}
    \caption*{(b) Slate 実装に限定した比較}
  \end{minipage}
  \caption{Append 操作のパフォーマンス比較。}
  \label{fig:append_performance}
\end{figure}

\subsection{Get Performance}

\subsubsection{理論値との整合性}

最初に、セクション \ref{sec:io_cost} で導出した I/O 回数の理論値 $\mathrm{IO}(n,i)$ が実測性能と整合することを確認する。図 \ref{fig:t352} に示す $T_{352}$ での理論的な I/O 回数分布に対し、図 \ref{fig:t352_exp} は $T_{352}$ での実測レイテンシである。理論分布では、最古のデータ ($i=0$) へのアクセスに最大 9 回の I/O 回数を要し、最新のデータ ($i=352$) へは $0$ 回 (メモリ上のエントリ $e_n$ のみ) でアクセス可能である。実測結果はこの理論分布との強い相関を示しており、$i=1$ で約 $0.015$ msec、$i=352$ で約 $0.003$ msec となっている。両分布に共通して観測される段階的な減少パターンと、各層別木の境界 ($i=256,320,352$) あるいはさらにその部分木の境界 ($i=64,128,192,\ldots$) での不連続な変化は、式 (\ref{eq:io}) で導出した I/O 回数と、式 (\ref{eq:io_bounds}) で定義した包絡線 $c^+(n,i)$ および $c^-(n,i)$ の構造を反映している。この結果は、Slate の Get 操作の性能がハミング距離に基づく I/O 回数によって支配されていることを実証している。

\begin{figure}
    \centering
    \includegraphics[width=0.75\linewidth]{figures/20260109000114-io.pdf}
    \caption{$T_{352}$ における各位置の Get レイテンシ分布。各点は位置 $i$ の葉ノードのデータへのアクセスに要した平均レイテンシを表す。$i=1$ (最古) から $i=352$ (最新) に向かう全体的な減少傾向と、層別木の境界における段階的変化は、図 \ref{fig:t352} に示した理論的 I/O 回数分布 $\mathrm{IO}(n,i)$ と対応している。}
    \label{fig:t352_exp}
\end{figure}

\subsubsection{大規模データでの比較評価} \label{sec:get_8m_performance}

図 \ref{fig:get_performance} に、$n=8\times 10^6$ のデータ構造に対して、最新データからの距離 $\delta=n-i+1$ を変化させたときの Get 操作のレイテンシを示す。横軸は対数スケールである。なお、各 Slate 実装の計測では各距離 $\delta$ において I/O 回数が上限 $c^+(n,i)$ に達する位置、すはわち最悪ケースとなるインデックスを意図的に選択している。したがって、実際のワークロードにおける平均レイテンシはここで報告する値より低くなる。

\textbf{定数時間アクセスの実装}. DoltDB、IAVL+、Binary Merkle Tree はいずれも距離 $\delta$ によらずほぼ一定のレイテンシを示す。ただし DoltDB は約 $0.38$ msec と最も遅い。Binary Merkle Tree は約 0.045 msec、IAVL+ は約 0.007 msec で安定しており、どちらも平衡木構造によって距離 $\delta$ によらず均一なアクセス特性を反映している。IAVL+ のレイテンシは下層の LevelDB のキャッシュが機能している可能性がある。

\textbf{Slate の非対称アクセス特性}. Slate (file) は最新データ近傍 ($\delta < 10$) で約 $0.003$ msec と最も高速であり、$\delta$ の増加に伴い対数的に増加して $\delta = 10^7$ で約 $0.08$ msec に到達する。この値は最悪ケースを計測しているため、同じ距離 $\delta$ における平均的なアクセスではより低いレイテンシを期待できる。この特性はセクション \ref{sec:io_cost} で導出した I/O 回数の理論値 $\mathrm{IO}(n,i)=O(\log(n-i+1))$ と整合する。曲線に観測される鋸歯状パターンは、ハミング距離に基づく I/O 回数の離散分布 (式 (\ref{eq:io})) を反映している。

\textbf{交差点の分析}. Slate (file) は最悪ケースにおいても $\delta < 10^3$ で Binary Merkle Tree より高速であった。平均ケースではこの交差はさらに右 (より大きな $\delta$) に移動すると予想される。これは、アクセスの大部分が最新近傍に集中するような時間局所性を持つ現実的なワークロードにおいてより大きな優位性を持つ。

\textbf{Slate (rocksdb) の特異な挙動}. Slate (rocksdb) の性能は $\delta < 10^5$ で Slate (file) より優位だが、それ以降で急激に悪化し、最古のデータで約 $0.33$ msec に達した。これは RocksDB の内部キャッシュが最新データに偏っており、古いデータへのアクセスで LSM-Tree の複数のレベルを走査する必要があるためと考えられる。


\begin{figure}[t]
    \centering
    \includegraphics[width=0.8\linewidth]{figures/20250920202203-get.pdf}
    \caption{Get 操作のパフォーマンス比較。}
    \label{fig:get_performance}
\end{figure}

\subsubsection{時間局所性のあるワークロード}

セクション \ref{sec:get_8m_performance} では位置ごとの Get レイテンシを評価したが、現実のワークロードでは最新データへのアクセスが偏る時間局所性を持つことが多い。本セクションでは Zipf 分布に従うアクセスパターンにおける Slate の性能特性を評価する。

\textbf{実験設定}. $n=2\times 10^9$ のデータ構造 $T_{2\mathrm{Gi}}$ に対して、最新データからの距離 $\delta=n-i+1$ が Zipf 分布 $p(\delta)\propto 1/k^s$ に従うようにアクセス位置を生成した。 歪度パラメータ $s \in \{0.5,1.2,1.5,2.0\}$ の 4 条件で各 $500$ 回のアクセスを行い、Get 操作にかかったレイテンシ $t$ の分布を計測した。$s$ が大きいほど最新データへのアクセスが集中する。

\textbf{結果}. 図 \ref{fig:get_zipf_distribution} に各条件での Get レイテンシ $t$ のヒストグラムを示す。歪度 $s$ の増加に伴い、レイテンシの分布は低レイテンシ側に集中する明確な傾向が観測された。

$s=0.5$ (弱い時間局所性) では、平均レイテンシ $\mu=0.0545$ msec、標準偏差 $\sigma=0.0124$ msec であり、分布は $0.05-0.06$ msec 付近にピークを持つ。これはアクセスが広範囲に分散し、古いデータへのアクセスも頻繁に発生することを反映している。

$s=1.5$ では $\mu=0.0113$ msec、$\sigma=0.0084$ msec となり、分布は $0.008$ msec 付近に集中する。$s=2.0$ (強い時間局所性) では $\mu=0.0086$ msec、$\sigma=0.0042$ msec と最も低く、アクセスの大部分が最新近傍で完結している。

\textbf{考察}. $s=0.5$ から $s=2.0$ への歪度の増加により、平均レイテンシは約 6.3 倍改善した。これは Slate の非対称アクセス特性が時間局所性を持つワークロードで有効に機能することを実証している。Web リクエストやブロックチェーントランザクションなど、現実のワークロードでは $s \ge 1.0$ 程度の時間局所性が報告されており \cite{almeida1996}\cite{barford1998}\cite{sornette2025}、Slate はこのような環境で Binary Merkle Tree や IAVL+ と比較して優位性を持つ。

\begin{figure}
    \centering
    \includegraphics[width=0.75\linewidth]{figures/20250918001917-biased-get-slate-file_y.pdf}
    \caption{$T_{2\mathrm{Gi}}$ に対する Zipf 分布に従うアクセスパターンにおける Get レイテンシの分布。歪度パラメータ $s$ の増加に伴い、レイテンシ分布が低レイテンシ側に集中する。各条件で $500$ 回のアクセスを計測した。}
    \label{fig:get_zipf_distribution}
\end{figure}

\subsection{Difference Detection Performance}

セクション \ref{sec:diff_detect_protocol} で定義した差異検出プロトコルの実測性能を評価する。この実験では、2 つの Slate (file) 木構造 $T^A_n$ と $T^B_n$ ($n=8\times 10^6$) を構築し、$d^A_i\ne d^B_i$ となるように位置 $i$ の値のみを異なる値に設定した。その後、プロトコルを開始して層サンプルの交換により差異位置 $i$ を特定するまでの時間を計測した。Get 操作と同様に、各距離 $\delta=n-i+1$ において最悪ケースとなる位置 ($\mathrm{IO}(n,i)=c^+(n,i)$ となる $i$) を選択した。

\textbf{結果}. 図 \ref{fig:diff_detect_performance} に差異検出時間と最新データからの距離 $\delta$ の関係を示す。検出時間は $\delta$ に対して対数的に増加しており、$\delta=1$ (最新の近傍) で約 $0.2$ msec、$\delta=10^7$ (最古の近傍) で約 $4.0\text{-}4.5$ msec を要した。

\textbf{理論値との整合}. 定理 \ref{theorem:diffdetect_finiteness} より、差異検出に要するラウンド数は最良ケースで $1$ 回、最悪ケースで $\lceil\log_2 n\rfloor=23$ 回である。図 \ref{fig:diff_detect_performance} の対数的増加傾向はこの理論的予想と整合している。

\textbf{従来手法との比較}. Raft のログ同期や Git の分岐点検出では、末尾からの線形探索により最悪ケースで $O(n)$ の比較が必要となる。$n=8\times 10^6$ では数秒から数十秒を要する可能性があるのに対し、Slate は最悪ケースでも約 $4.5$ msec で差異位置を特定できる。これは 3-4 桁の性能改善に相当する。

\textbf{最新近傍での高速検出}. 現在の分散システムでは、レプリカ間の差異位置は最新近傍に集中することが多い。ネットワーク遅延や一時的な障害による分岐は、通常、直近の数エントリから数百エントリの範囲で発生する。$\delta<100$ の範囲では検出時間が $0.6$ msec 以下であり、このような典型的なケースでは極めて高速な分岐検出が可能である。

\textbf{考察}. これらの結果は Slate の差異検出プロトコルが大規模ログに対して効率的に機能することを実証している。特に、追記偏重ワークロードにおいて分岐が最新近傍に集中する場合、Slate は従来手法と比較して大幅な性能向上を達成する。

\begin{figure}
    \centering
    \includegraphics[width=0.75\linewidth]{figures/20250918001917-diffdetect.pdf}
    \caption{$T_{8\mathrm{Mi}}$ の木構造において分岐位置の検出に要した時間。位置 $i$ の葉のみを異なる値にした 2 つの $T_{8\mathrm{Mi}}$ の木構造から、差異検出プロトコル (セクション \ref{sec:diff_detect_protocol}) を実行して分岐位置を特定するまでの平均時間を示す。横軸は最新データからの距離 $\delta=n-i+1$ を対数スケールで表示している。角距離において最悪ケースとなる位置を選択した。}
    \label{fig:diff_detect_performance}
\end{figure}

\subsection{Cache Effectiveness} \label{sec:cache_effectiveness}

セクション \ref{sec:io_cost} で示したように、Get 操作の I/O 回数 $\mathrm{IO}(n,i)$ は最新エントリ $e_n$ から目的の葉 $b_i$ への経路上の左枝移動回数に等しい。したがって、木構造の上位に位置するエントリを追加でメモリ上にキャッシュすることで I/O 回数を削減できる。

本実装では、最新の $e_n$ から左枝方向に $k$ ホップ以内で到達可能なエントリの集合を\textbf{レベル $k$ キャッシュ}と定義する。レベル $0$ は本論文で提案するように最新エントリ $e_n$ のみを保持し、空間計算量 $O(\log n)$ ですべての操作で 1 回の I/O を削減する。レベルが 1 つ上がるごとに、すべての操作で 1 回の I/O が削減される。レベル $k$ キャッシュは最大 $2^k$ 個のエントリを保持して空間計算量 $O(2^k \log n)$ となり、Get 操作の I/O 回数を最大 $k$ 回削減する。ただし、Append 操作ごとにキャッシュの更新が必要となり、更新コストはレベルに応じて増加する。具体的には、新しいエントリの追加と、ホップ数が $k$ を超えたエントリの追い出しが必要となる。

Slate の構造的効率性により、キャッシュによる追加の性能向上は限定的である可能性がある。例えば $n==2^{64}$ という極めて大規模データセットに対しても、最悪ケースは $\mathrm{IO}(2^{64},1)=64$ 回に過ぎない。さらに、時間的局所性の強いワークロードでは式 (\ref{eq:io}) より実際の I/O 回数は大幅に小さくなる。したがって、多くの用途ではレベル 0 キャッシュのみで十分な性能が得られるだろう。ただし、Get の頻度が著しく高いケースや、時間的局所性が低くアクセスが昨日名ケース、1 回の I/O コストが著しく高いワークロードにおいてはレベル 1 以上のキャッシュが有効となるケースもある。

このセクションでは、キャッシュレベルを変化させたときの Get 性能への影響を評価する。

\textbf{実験設定}. $T_{8\mathrm{Mi}}$ に対して、$\mathrm{IO}(n,i)=c^+(n,i)$ の最悪ケースとなる位置 $i$ を選択し、キャッシュレベル $L\in\{0,1,2,3\}$ の 4 条件で Get 性能を計測した。Level 0 は最新エントリ $e_n$ のみをメモリ上に保存する通常の構成であり、Level $k$ は $e_n$ から左枝方向に $k$ ホップ以内で到達可能なすべてのエントリをキャッシュする。キャッシュされるエントリ数は Level $k$ に対して最大 $2^k$ 個である。

\textbf{結果}. 図 \ref{fig:cache_level_comparison} にキャッシュレベル別の Get レイテンシを示す。全条件で対数的増加傾向は維持され、I/O 回数に対応する段階的な曲線が見られるが、キャッシュレベルの増加に伴い曲線が下方にシフトしている。

最新近傍での全条件は約 $0.0003$ msec である。これは、キャッシュ内のエントリのみでアクセスが完結し I/O 回数が $0$ で完結するためである。 

\begin{figure}
    \centering
    \includegraphics[width=0.75\linewidth]{figures/20250918001917-cache.pdf}
    \caption{$T_{8\mathrm{Mi}}$ におけるキャッシュレベルによる Get 性能。}
    \label{fig:cache_level_comparison}
\end{figure}

\section{Discussion}

本セクションでは、実験結果の解釈、他のデータ構造との比較分析、適用可能なユースケース、設計上のトレードオフ、および制限事項について議論する。

\subsection{結果の解釈}

実験結果は、Slate の背系原理が追記偏重ワークロードにおいて有効であることを実証している。

\textbf{理論と実測の整合性}. セクション \ref{sec:io_cost} で導出した I/O 回数の理論値 $\mathrm{IO}(n,i)$ は、図 \ref{fig:t352_exp} に示す実測レイテンシと強い相関を示した。特に、層別木の境界における段階的な変化パターンは、式 (\ref{eq:io}) で定義したハミング距離に基づく I/O 回数分布を正確に反映している。ストレージ消費量についても、$n=8\mathrm{Mi}$ において理論値 $5.0\mathrm{GiB}$ に対して実測値約 $5.4\mathrm{GB}$ と整合している。

\textbf{非対称アクセス特性の有効性}. 図 \ref{fig:get_zipf_distribution} に示す Zipf 分布に従うアクセスパターンの実験では、歪度パラメータ $s$ の増加に伴い平均レイテンシが約 $6.3$ 倍改善した。これは、Slate の非対称構造が時間局所性を持つ現実的なワークロードにおいて機能することを示している。Web リクエストやブロックチェーントランザクションでは $s \ge 1.0$ 程度の時間局所性が報告されており、Slate はこのような環境で顕著な優位性を持つ。

\textbf{追記性能の優位性}. Slate (file) は IAVL+ と比較して約 56 倍、DoltDB と比較して約 66 倍の追記スループットを達成した。この性能差は、Slate の直列化形式が単一のシーケンシャルファイルの追記で完結する設計に起因する。IAVL+ では追記操作ごとに AVL 木の再平衡化と LevelDB への複数回の書き込みが発生し、DoltDB では Prolly Tree の確率的分割とインデックス更新のオーバーヘッドが加わる。一方、命題 \ref{prop:append} より、Slate では新しいエントリ $e_n$ の構築に必要な情報がすべて直前のエントリ $e_{n-1}$ に含まれているため、追加の読み込み I/O なしに追記を完了できる。

\subsubsection{他のデータ構造との比較分析}

\textbf{History Tree との関係}. Slate は History Tree を基盤としていくつかの重要な拡張を行っている。第一に、エントリ単位での直列化形式を定義し、追記性能が $O(1)$ の書き込み I/O で完結することを保証した。History Tree の論文では直列化形式 (保存形式) には議論されておらず、したがって I/O 効率についても議論されていない。第二に、セクション \ref{sec:diff_detect_protocol} 分岐検出プロトコルを設計し、複数のログ間の差異位置を $O(\log^2 n)$ の通信複雑度で特定する手法を確立した。第三に、式 (\ref{eq:io_bounds}) に示すように時間局所性に基づく I/O 回数の上限・下限を数学的に導出し、アクセスパターンと性能の関係を形式化した。

\textbf{平衡木ベースの認証付きデータ構造との比較}. IAVL+ や従来の平衡 Merkle Tree は、任意のキーに対して均一な $O(\log n)$ アクセスを提供する。この特性は前提条件のない操作には適しているが、その平坦な設計は追記偏重かつ時間局所性を持つワークロードでは標準的なケースの効率を落としている。Slate は追記最適化という制約を受け入れることで、最新データへの $O(1)$ アクセスと、最悪ケースでも $O(\log n)$ を維持する非対称特性を実現している。図 \ref{fig:get_performance} に示す実験結果では、$\delta < 10^3$ の範囲で Slate の最悪ケースが Binary Merkle Tree より高速であり、時間局所性のあるワークロードではさらに有利な範囲が拡大することが予想される。

\textbf{Prolly Tree との比較}. DoltDB で使用される Prolly Tree は、コンテンツ依存の確率的分割により構造的な差分検出を可能にする。これは Git のような完全な分岐・マージ履歴をサポートするバージョン管理システムに向いている。一方、Slate は線形の履歴のみを想定し、二つの履歴の差異 (分岐) を効率的に検出することを目的としている。この設計上の選択により、Slate は Prolly Tree は Slate と比較して約 66 倍の追記性能を達成しているが、非線形な履歴管理は本質的にサポートしていない。

\subsubsection{適用可能なユースケース}

\textbf{分散トランザクションログ}. Kafka や Raft のような分散ログシステムでは、レプリカ間の同期と差異の修復が頻繁に発生する。従来の手法では末尾からの線形添削で分岐点を検出するため、大規模な履歴遡及が必要な場合に $O(n)$ の比較コストが発生する。Slate の差異検出プロトコルはこれを $O(\log n)$ ラウンドに改善し、各ラウンドで $O(\log n)$ のハッシュ値のみを交換する。図 \ref{fig:diff_detect_performance} に示す実験結果では、$n=8\mathrm{Mi}$ のデータ列において、最悪ケースであっても約 $4.5$ msec で差異 (分岐) 位置を特定できることが示されている。

\textbf{ブロックチェーン}. ブロックチェーンの軽量クライアントは完全なブロック履歴を保持せずに特定のトランザクションの存在を証明する必要がある。Slate の層サンプルは $O(\log n)$ サイズで従来の Merkle Tree と同等の包含証明を提供する。加えて、Slate の時間局所性により、最新ブロックの検証は $O(1)$ で完了する。また、Merkle Tree の特性を持つ Slate はデータ列全体を証明する構造として利用できることから、ブロック列またはトランザクション列そのものを Slate で構成するブロックチェーンも可能である。

\textbf{非同期レプリケーション}. 地理的に分散したデータセンター間の非同期レプリケーションでは、ネットワーク分断後の再同期が課題となる。Slate は同期再開時に分岐が発生していないことを暗号論的に証明し、もし破損が存在する場合は必要な部分の迅速な転送と修復を可能にすることで回復時間目標 (RTO) の短縮に寄与する。

\textbf{監査ログと改ざん検出}. Certificate Transparency のような監査ログシステムでは、ログの完全性検証と効率的な監査が求められる。Slate は History Tree と同等の暗号論的保障を提供しながら、追記と最新データアクセスの効率を改善している。

\subsubsection{設計上のトレードオフ}

\textbf{空間効率 vs. 認証機能}. Slate の空間計算量 $O(n \log n)$ は、木構造を持たない $O(n)$ の単純なシーケンシャルファイルと比較して $\log n$ 倍のオーバーヘッドを伴う。理論値では 8 バイトのデータを保存する $n=8\mathrm{Mi}$ の Slate (file) のストレージ消費量 $|S_{8\mathrm{Mi}}|=5.0\mathrm{GiB}$ に対し、木構造 (中間ノード) に必要な領域は $4.5\mathrm{GiB}$ であり、これは全体の約 $90$\% を占めている。このオーバーヘッドは、分岐検出と完全性検証の機能を提供するために必要なトレードオフである。

\textbf{最新変調 vs. 均一アクセス}. Slate は最新データへのアクセスを $O(1)$ で提供する代わりに、古いデータへのアクセスコストが対数的に増加する。図 \ref{fig:get_performance} が示すように、$\delta$ が大きい範囲では Binary Merkle Tree や IAVL+ の方が低レイテンシを達成する。したがって、アクセスパターンが均一なワークロード (例えばランダムアクセスや監査目的での全履歴スキャン) では Slate の優位性は限定的となる。

\textbf{追記専用 vs. 汎用操作}. Slate は追記と切り詰めのみをサポートし、任意位置への挿入や個別要素の更新をサポートしていない。この制約により直列化形式の単純性と追記性能が達成されているが、一般的なデータベースのワークロードには適さない。

\textbf{メモリフットプリント}. Slate の効率的な動作には、最新エントリ $e_n$ をメモリ上に保持することが前提となる。$e_n$ のサイズは $O(\log n)$ であり、$n=2^{64}$ 付近でも高々 $64$ 個のノード (各ノード 50 bytes として約 3.2 kB) に過ぎないため、実用上の制約とはならない。しかし、セクション \ref{sec:cache_effectiveness} で述べたレベル $k$ キャッシュを使用する場合、$O(2^k \log n)$ のメモリが必要となり、メモリ制約のある環境では考慮が必要である。

\subsubsection{制限事項}

\textbf{ランダムアクセス性能}. Slate は時間局所性を前提として設計されており、一様にランダムなアクセスでは Binary Merkle Tree や IAVL+ に劣る場合がある。

\textbf{非線形履歴}. 本論文で提案する Slate は線形のログ構造を前提としており、Git のような分岐と統合を伴う非線形な履歴はサポートしない。ただし、Slate の永続性を利用して、木構造の世代を分岐に対応することは可能であると考えられる。

\textbf{切り詰めと並行性}. セクション \ref{sec:truncate_operation} で議論したように、Truncate 操作は並行読み取りと競合する可能性がある。本論文では SWMR モデルにおける Append と Get の並行実行の安全性を示したが、Truncate を含む完全な並行制御については、コミット済み領域の保護や MVCC の実装など、追加の機構が必要となる。

\textbf{ネットワークレイテンシ}. セクション \ref{sec:diff_detect_protocol} で議論した差異検出プロトコルは、最悪ケースで $\lceil\log_2 n\rceil$ ラウンドの通信を要する。各ラウンドで $O(\log n)$ サイズの層サンプルを交換するため、通信量自体は小さいが、高レイテンシなネットワーク環境では往復遅延が累積する。差異位置の事前予測が可能な場合は、初期の層サンプルの選択を最適化することで改善の余地がある。

\subsubsection{将来の拡張可能性}

本論文で提示した Slate の設計は、追記偏重かつ時間局所性を持つワークロードに特化しており、上述の制限事項はこの設計選択の帰結である。以下の拡張は特定のユースケースにおいて有用となり得るが、これらは Slate の中核的な貢献とは独立した最適化であり、本論文の範囲外とする。参照実装は http://github.com/torao/slate で公開されてており、実務者による拡張や応用を歓迎する。

\textbf{Compaction via Algebraic State Aggregation}. Slate の各葉ノードが結合可能な状態を表す場合、各層別木は LSM-Tree に似た段階的コンパクション機構を実装できる。具体的には、状態集合 $S$ 上の結合的な二項演算子 $\oplus: S\times S\to S$ が定義されているとき、各層別木 $T'_{i,j}$ に含まれる $2^j$ 個の葉ノードの状態を $\oplus$ で集約した単一の状態 $s'_{i,j}=s_{i-2^j+1} \oplus s_{i-2^j+2} \oplus \ldots \oplus s_i$ に置換できる。この集約により、完全二分木として確定した層別木は個々の葉を保持する必要がなくなり、空間計算量を $O(n \log n)$ から $O(\log n)$ に削減できる。演算子 $\oplus$ が可換かつ冪等である場合 (例: 集合の和、最大値、ブルームフィルタの合成など)、集約順序に依存しない決定論的な結果が保証される。このような手法は、イベントソーシングにおける状態スナップショット、時系列データベースにおけるダウンサンプリング、または分散カウンタの集約といった目的で応用されており、Slate の追記最適化特性を維持しながら状態保持効率を改善する拡張として検討することができる。

\textbf{適応的キャッシュ戦略}. セクション \ref{sec:cache_effectiveness} で評価したレベル $k$ キャッシュは静的な構成であるが、利用可能なメモリリソースとワークロードの時間局所性を動的に推定し、キャッシュレベル $k$ を適用的に調整したり、非均等ホップ数のキャッシュ、あるいはアクセス過多の層別木ルートのキャッシュなど、さまざまな戦略が考えられる。特に、アクセスパターンが時間とともに変化する環境では、適応的なキャッシュ戦略がメモリ効率と性能のバランスを改善する可能性がある。

\textbf{並列差異検出}. 現在の差異検出プロトコルは単一の層サンプルを逐次的に交換するが、複数の部分木に対して並行して差異検出を実行することでラウンド数を削減できる可能性がある。特に、複数の差異位置を検出する場合や、差異位置の分布に事前知識がある場合に有効となり得る。

\section{Related Work}

本セクションでは Slate に関連する先行研究を概観し、本研究の位置づけを明確にする。関連研究は Merkle Tree とその変種、認証付きデータ構造、永続データ構造、ログ構造ストレージ、および分散システムにおけるログ同期の 5 つの観点から整理する。

\subsection{Merkle Tree とその変種}

\textbf{Merkle Tree}. Merkle \cite{merkle1989} が提案したハッシュツリーは、データ集合の完全性を効率的に検証するための技術基盤である。完全二分木構造において、各葉ノードはデータ要素のハッシュ値を保持し、各中間ノードは子ノードのハッシュ値の連結から計算されるハッシュ値を保持する。ルートハッシュは木全体のダイジェストとして機能し、$O(\log n)$ サイズの包含証明 (所属証明または認証パス) により任意の要素の存在を検証できる。しかし、古典的な Merkle Tree は静的なデータ集合を前提としており、動的な追加操作に対しては木全体の再構築が必要となる。

\textbf{History Tree}. Crosby と Wallach \cite{crosby2009} は、追記操作をサポートする動的な Merkle Tree の変種である History Tree を提案した。この構造は $n$ の 2 進数表現に対応する完全二分部分木の集合として木を構成し、新しいデータの追加時に $O(\log n)$ のノードのみを更新する。Slate は History Tree の構造的特性を継承しつつ、直列化形式の定義、I/O 複雑度の形式化、および分岐(差異)検出プロトコルの設計という拡張を行っている。特に、History Tree の論文では二次記憶装置への永続化や分散環境でのログ同期については議論されておらず、本論文はこれらの実用的側面を補完している。

\textbf{Merkle Mountain Range}. Todd が提案した Merkle Mountain Tree (MMR) \cite{todd2013MMR} は History Tree と類似した追記専用の構造を持つ。MMR は "Mountain" と呼ばれる複数の完全二分木 (Slate での層別木) の列として表現され、各追加操作で高々 $O(\log n)$ 個のノードが更新される。これは Grin や Mimblewimble プロトコルなどのブロックチェーン実装で採用されている。最新の研究により、MMR は Witness 更新回数において漸近的に最適であることが証明されている \cite{bonneau2025merkle}。Slate と MMR は構造的に類似しているが、Slate はエントリ単位での直列化形式を明示的に定義し、時間局所性に基づく I/O 回数の理論的解析を提供している点で異なる。

\textbf{Merkle Search Tree}. Auvolat と Taïani はキー順序付きデータに対する確率的な Merkle Tree である Merkle Search Tree \cite{auvolat2019merkle} を提案した。この構造は、キーのハッシュ値に基づいて木の形状を決定し、同一のキー集合に対して決定論的な木構造を生成する。Slate との主要な違いは、Merkle Search Tree がキー順序に基づく任意の位置への挿入をサポートするのに対し、Slate は時間順序に基づく追記操作に特化している点である。この設計選択により、Slate は追記操作の I/O を $O(1)$ に抑えている。

\textbf{Prolly Tree}. Prolly Tree (Probabilistic B-Tree) は Noms プロジェクトで開発された content-addressed な B-Tree の変種である \cite{Noms2015ProllyTree}。同等の構造は Merkle Search Tree \cite{auvolat2019merkle} として独立に提案されている。これらはローリングハッシュを用いて確率的にノード境界を決定することで、同一のデータ集合に対して決定論的な木構造が生成される。Prolly Tree を用いた DoltDB では Git のような分岐と統合を効率的にサポートする。Slate は線形な履歴のみを想定し、非線形な分岐管理をサポートしないが、代わりに追記性能で約 66 倍の優位性を達成している。

\textbf{Amazon Dynamo}. Merkle Tree を用いた差異検出は、Amazon Dynamo の Anti-Entropy プロトコル \cite{decandia2007dynamo} 以降、分散 Key-Value ストアにおけるレプリカ同期で広く採用されている。これらのシステムは、2 つの状態セット間の「どのキー範囲が異なるか」を検出する。一方、Slate が扱う問題は、2 つの時系列ログ間の「いつ分岐したか」という分岐点の位置特定である。この問題設定の違いにより、Slate は append-only の成長パターンと時間的局所性を活用した最適化が可能となる。

\subsubsection{認証付きデータ構造}

\textbf{IAVL+}. Cosmos SDK で使用される IAVL+ \cite{kwon2016cosmos} \cite{CosmosIAVL} は、AVL 木に Merkle Tree ハッシュを組み合わせた認証付きデータ構造である。各ノードは部分木のハッシュ値を保持し、任意のキーに対する包含証明を $O(\log n)$ で生成できる。また IAVL+ は各バージョンを永続化し、過去の状態へのアクセスを可能にする。しかし、各更新操作で AVL 木の再平衡化が発生するため、追記偏重ワークロードでは Slate と比較して約 56 倍のオーバーヘッドが生じる。

\textbf{Authenticated Data Structures}. Tamassia は認証付きデータ構造 \cite{tamassia2003auth} の一般的な枠組みを提示した。この枠組みでは、信頼できるソースがデータ構造のダイジェストを公開し、信頼できないサーバがクエリ結果と共に検証可能な証明を提供するという、Merkle Tree を一般化している。Slate はこの枠組みにおいて、追記専用ログに特化した認証付きデータ構造として位置づけられる。あるいは、ノードに注目すれば木構造だが、エントリに注目すれば認証付きスキップリストとも見える。定義 \ref{def:stratum_sample} の層サンプルは包含証明の変種であり、差異検出プロトコルは 2 つの認証付きログ間の一貫性検証のみならず、その差異の位置の特定に拡張している。

\textbf{Certificate Transparency}. Laurie らが設計した Certificate Transparency (CT) \cite{ct2021} は、TLS 証明書の発行履歴を公開監査可能なログとして記録するシステムである。CT ログは History Tree (RFC 6962 では Merkle Tree Hash と呼称) を使用し、証明書の包含証明と一貫性証明を提供する。Slate は CT ログと類似した構造を持つが、分散レプリカ間の分岐検出と修復という、CT では想定されていないユースケースに焦点を当てている。CT ログは単一の信頼できるログサーバを前提としているのに対し、Slate は複数のレプリカ間での不整合を許容し、その効率的な検出と修復を目的としている。

\textbf{Verifiable Data Structures}. Verifiable Data Structure \cite{Eijdenberg2015VDS} は Certificate Transparency の概念を一般化して検証可能なログモード (追記専用) とマップモード (Key-Value) をサポートし、さまざまなアプリケーションに対応する。これらは Google の Trillian \cite{GoogleTrillian} に実装されている。Slate の直列化形式と分岐検出プロトコルは、時間局所性を活用した非対象アクセス効率という独自の最適化を提供する点で異なる。

\subsubsection{永続データ構造}

\textbf{Partially Persistent Data Structures}. Driscoll ら \cite{driscoll1986} は、データ構造の永続化に関する体系的な手法を提示した。部分的永続化 (partial persistence) では、すべての過去のバージョンへの読み取りアクセスと、最新バージョンへの更新が可能となる。彼らの手法は、各ノードにバージョンスタンプと追加のポインタを付与することで、任意のデータ構造を永続化する。Slate は、部分的永続化の特殊ケースとして、2 進数表現に基づく自然なバージョン階層を活用している。各エントリ $e_i$ は世代 $i$ で追加されるすべてのノードを集約しており、Driscoll らの手法で必要とされる複雑なポインタ管理やバージョンスタンプを排除している。

\textbf{Functional Data Structures}. 関数型プログラミングの文脈では、不変データ構造 (immutable data structures) が広く研究されている。Okasaki \cite{okasaki1998purely} は永続的なリスト、木、キューなどの効率的な実装を体系化した。Slate の層別木は不変性を持ち、一度作成された完全二分部分木は将来の操作で変更されることがない。この特性により、セクション \ref{sec:concurrent_access_model} で示した SWMR 並行実行モデルにおける安全な並行アクセスが保証される。

\textbf{Copy-on-Write B-Tree}. Rodeh は、ファイルシステム向けの Copy-on-Write B-Tree \cite{rodeh2008b} を提案した。この構造は、更新時に影響を受けるノードのみをコピーし、過去のバージョンを効率的に保持する。Btrfs や ZFS などの現代的なファイルシステムで採用されている。Slate のエントリ構造も類似した原理に基づいているが、B-Tree の汎用的な挿入・削除操作ではなく、追記専用操作に特化することで、より単純かつ効率的な実装を実現している。

\subsubsection{ログ構造ストレージ}

\textbf{Log-Structured Merge Tree}. O'Neil らが提案した LSM-Tree \cite{o1996log} は書き込み最適化されたインデックス構造である。メモリ上のバッファに書き込みを蓄積し、定期的にディスク上のソート済みファイルにマージする。LevelDB、RocksDB、Cassandra などの多くの現代的なストレージシステムで採用されている。Slate は LSM-Tree とは異なるアプローチを取り、各エントリを独立した追記単位として直列化する。これにより、LSM-Tree で必要なコンパクション処理を排除し、追記操作を $O(1)$ の I/O で完結させている。図 \ref{fig:append_performance} では Slate (file) が Slate (rocksdb) と比較して 3 倍以上の追記性能を達成しており、これはコンパクションオーバーヘッドの排除による効果であると考えられる。

\textbf{Write-Ahead Logging}. データベースシステムにおける Write-Ahead Log (WAL) \cite{mohan1992aries} は、トランザクションの永続性を保証するための標準的な手法である。WAL は追記専用のログ構造であり、Slate の想定するワークロードと親和性が高い。従来の WAL はインデックスを持たないシーケンシャルファイルであるが、Slate は WAL に分岐検出機能と効率的なランダムアクセスを付与する拡張として位置づけられる。

\textbf{Kafka.} Kreps ら \cite{kafka2011} が開発した Apache Kafka は、分散メッセージングシステムとして広く利用されている。Kafka のログは追記専用のパーティション化された構造を持ち、オフセットベースの消費を提供する。Kafka のレプリカ間同期は、リーダーからフォロワーへの線形名複製に基づいており、分岐検出の効率化は本研究の主要な同期の一つである。Slate の差異検出プロトコルは Kafka のような分散ログシステムにおけるレプリカ間同期の効率を改善する可能性がある。

\subsubsection{分散システムにおけるログ同期}

\textbf{Raft}. Raft \cite{ongaro2014raft} は分散システムにおける合意プロトコルの代表的なアルゴリズムの一つである。Raft のログ複製は、リーダーがフォロワーに AppendEntries RPC を送信し、フォロワーは自身のログとの整合性を確認する。ログの不整合が検出された場合、Raft はバックトラッキングにより不一致点を探索するが、最悪ケースでは $O(n)$ の比較が必要となる。Slate の差異検出プロトコルはこの探索を $O(\log n)$ ラウンドに改善する。ただし、Slate はログの内容の暗号論的比較を前提としており、Raft の term と index に基づく比較とは異なるアプローチである。

\textbf{Vector Clock と Version Vector}. 分散システムにおける因果関係の追跡には Vector Clock \cite{mattern1988virtual} や Version Vector \cite{parker1983detection} が広く使用されている。これらの手法は、イベント間の happened-before 関係を効率的に判定する。Slate は因果関係の追跡ではなく、2 つのログ間の内容の同一性を暗号論的に検証することに焦点を当てている。したがって Vector Clock と Slate は補完的な技術であり、組み合わせて使用することが可能である。

\textbf{Merkle-CRDTs}. Sanjuan らは Merkle DAG と Conflict-free Replicated Data Types (CRDTS) \cite{shapiro2011conflict} を組み合わせた Merkle-CRDTs \cite{sanjuan2020merkle} を提案した。この構造は、分散環境での効率的な同期と、操作の交換可能性を保証する。IPFS や libp2p などの分散システムで採用されている。Slate は CRDT の特性を持たないが、追記専用と制約により、操作の順序に関する競合が発生しない単純なモデルを提供している。

\textbf{Anti-Entropy Protocols}. Demers らが提案した Anti-Entropy プロトコル \cite{demers1987epidemic} は、分散システムにおけるレプリカ間の不整合を検出・修復するための一般的な手法である。Merkle Tree を用いた効率的な差分検出は、Amazon Dynamo \cite{decandia2007dynamo} や Cassandra \cite{lakshman2010cassandra}\cite{van2008efficient}、Riak \cite{preguicca2010dotted} などの分散データベースで採用されている。Slate の差異検出プロトコルはこれらの手法を拡張し、追記専用ログに特化した最適化を行っている。特に、時間局所性に基づく非対称な I/O 特性は、分岐が最新近傍に集中する典型的なケースでの効率を改善する。

\textbf{共通祖先問題}. 本研究が扱う分岐検出は、表面的には最小共通祖先 (LCA) 問題または最長共通接頭辞 (LCP) 問題と類似している。しかし、従来の LCA アルゴリズム \cite{harel1984fast} は静的なグラフ構造を前提としてインデックスを事前構築し、各ノードの構造的関係を高速に検索することを目的としている。一方、この論文で提案する分散システムにおける実際の問題は、異なるマシン上で独立に成長したデータ系列の整合性を暗号論的に検証しながら、双方の分岐点を遠隔で特定することである。Slate はこのような分散かつ動的な環境に特化して設計されており、従来の LCA 研究とは根本的に異なる問題設定を対象としている。

\subsubsection{時間局所性とキャッシュ}

\textbf{Web アクセスパターン}. Almeida ら \cite{almeida1996} および Barford と Crovella \cite{barford1998} は、Web リクエストにおけるアクセスパターンが重尾分布 (特に対数正規分布やべき乗則) に従うことを示した。この時間局所性は、キャッシュ設計やプリフェッチ戦略に重要な示唆を与える。Slate の設計は、このような時間局所性を前提とし、最新のデータへのアクセスを $O(1)$ 回の I/O で提供することで、現実的なワークロードでの効率を最適化している。

\textbf{ブロックチェーンのアクセスパターン}. Sornette と Zhang \cite{sornette2025} は、Bitcoin におけるトランザクションフロートと保持時間がべき乗則に従うことを示した。ブロックチェーンのような追記専用データ構造においても、最新データへのアクセスが統計的に偏ることが確認されている。Slate の時間局所性に基づく設計は、このようなワークロードに特に適している。

\subsubsection{本研究の位置づけ}

表 \ref{table:Comparison_of_characteristics_with_related_data_structures} に、Slate と関連するデータ構造の特性比較を示す。Slate は History Tree の構造を基盤としながら、エントリによって時間局所性を空間局所性にパックし、追記 I/O の $O(1)$ 化、時間局所性に基づくアクセス効率の形式化、および分散環境での分岐検出プロトコルという独自の貢献を行っている。既存の認証付きデータ構造 (IAVL+、Polly Tree) が汎用的な操作をサポートするのに対し、Slate は追記専用という制約を受け入れることで、追記偏重ワークロードにおける大幅な性能向上を達成している。

\begin{table}
    \centering
    \begin{tabular}{llllll}
         \hline
         データ構造 & 追記 I/O & Get I/O (最新) & Get I/O (最古) & 分岐検出 & 任意挿入 \\
         \hline \hline
         Binary Merkle Tree & $O(n)$ & $O(\log n)$ & $O(\log n)$ & $O(n)$ & - \\ \hline
         History Tree & $O(\log n)$ & $O(1)$ & $O(\log n)$ & - & - \\ \hline
         IAVL+ & $O(\log n)$ & $O(\log n)$) & $O(\log n)$ & - & ✔ \\ \hline
         Polly Tree & $O(\log n)$ & $O(\log n)$ & $O(\log n)$ & $O(\log n)$ & ✔ \\ \hline
         Merkle Search Tree & $O(\log n)$ & $O(\log n)$ & $O(\log n)$ & $O(\log n)$ & ✔ \\ \hline
         Slate & $O(1)$ & $O(1)$ & $O(\log n)$ & $O(\log^2 n)$ & - \\ \hline
    \end{tabular}
    \caption{関連データ構造との特性比較。I/O 複雑度は二次記憶装置への理論的なアクセス回数を示す。分岐検出は 2 つの木構造間の差異位置を特定するための複雑度を示す。"-" は該当する操作が定義されていないか、本研究で議論されていないことを示す。}
    \label{table:Comparison_of_characteristics_with_related_data_structures}
\end{table}

\section{Conclusion}

本論文は、分散ログの同期と分岐検出に特化した追記最適化 Merkle Tree である Slate (Stratified Hash Tree) を提案した。Slate は History Tree の構造を基盤としながら、追記偏重かつ時間局所性を持つワークロードに対して理論的・実践的な最適化を実現している。

\textbf{理論的貢献}. 本研究は、Slate における I/O 回数がハミング距離に従うことを形式化し、最新エントリから任意の葉への到達コストの上限 $c^+(n,i)$ および下限 $c^-(n,i)$ を導出した。この解析により、最新データへのアクセス効率が $O(1)$ と最適で、過去のデータへのアクセスでも $O(\log n)$ を超えないという非対称なアクセス特性が数学的に保証される。また、$n$ の 2 進数表現に基づく決定論的な木構造の構築手法を確立し、同一世代に対して常に同一の構造が生成されることを示した。

\textbf{設計上の貢献}. Slate の直列化形式は、各エントリが独立した追記単位として機能し、追記操作が $O(1)$ の Write I/O で完結することを保証する。命題 \ref{prop:append} で示したように、データ $d_n$ の追加に伴う新しいエントリ $e_n$ の構築に必要な情報はすべて直前のエントリ $e_{n-1}$ に含まれており、追加の Read I/O を要しない。この特性は LSM-Tree のようなコンパクション処理を必要とする既存手法と対照的である。さらに、層サンプルに基づく分岐検出プロトコルを設計し、2 つのレプリカ間の差異位置を $O(\log n)$ ラウンド、$O(\log^2 n)$ の通信複雑度で特定する手法を確立した。

\textbf{実験的検証}. Rust における参照実装を用いた評価では、Slate の設計原理が実測性能と整合することを確認した。追記性能において、Slate (file) は IAVL+ と比較して約 56 倍、DoltDB と比較して約 66 倍のスループットを達成した。参照性能では、理論的に導出した I/O 回数分布 $\mathrm{IO}(n,i)$ と実測レイテンシが強い相関を示し、Zipf 分布に基づくアクセスパターンでは歪度の増加に伴い平均レイテンシが約 6.3 倍改善した。分岐検出では $n=8\mathrm{Mi}$ における最悪ケースでも約 4.5 msec で差異位置を特定でき、従来の線形探索手法と比較して 3-4 桁の性能改善を達成した。ストレージ消費量についても、$n=8\mathrm{Mi}$ において理論値 $5.0\mathrm{GiB}$ に対して実測値約 $5.4\mathrm{GB}$ と整合し、理論解析の妥当性を実証した。

\textbf{実用的意義}. Slate は、Merkle Tree の適用範囲に加え、分散トランザクションログ、非同期レプリケーション、バージョン履歴管理システム、ブロックチェーンといった追記偏重かつ時間局所性を持つワークロードに適用可能である。特に、レプリカ間の分岐が最新近傍に集中する典型的なケースでは、Slate の非対称アクセス特性と効率的な分岐検出プロトコルが顕著な優位性をもたらす。追記専用という制約を受け入れることで、汎用的な認証付きデータ構造と比較して大幅な性能向上を達成しており、この設計選択は対象とするワークロードにおいて合理的なトレードオフである。

参照実装は https://github.com/torao/slate で公開されており、本論文で提示した理論と設計の検証、および実務での応用に利用可能である。

\newpage

\bibliographystyle{IEEEtran}  % IEEE 形式(番号式)
\bibliography{refs}            % 参照する .bib ファイル名(拡張子不要)

\appendix

\newpage
\section{ビット演算と CPU 命令への変換}

Slate の構造は二分木に基づいており、その数学的な表現において 2 のべき乗や余剰の演算が頻繁に現われる。これらは汎用プログラミング言語の多くがサポートする単純な四則演算、ビット演算、および CPU 命令を用いて高速に算出することができる。数式表記との対応表を Table \ref{table:cpu_instruction} に示す。

\begin{table}[h]
  \centering
    \begin{tabular}{ll}
     \hline
      \textbf{数式表記} & \textbf{ビット演算または CPU 命令} \\
      \hline \hline
      $2^j$ & \texttt{1 << j} \\ \hline
      $i \bmod 2^j$ & \texttt{i \& ((1 << j) - 1)} \\ \hline
      $i - (i \bmod 2^j)$ & \texttt{i \& \^((1 << j) - 1) = (i >> j) << j} \\ \hline
      $\left\lfloor \frac{i}{2^j} \right\rfloor$ & \texttt{i >> j} \\ \hline
      $\left\lceil \frac{i}{2^j} \right\rceil$ & \texttt{(1 + (1 << j) - 1) >> j} \\ \hline
      $\lfloor \log_2 x \rfloor$ & \texttt{ilog2(x)}: $x$ の 2 進数表現で最も左にある 1 の位置。 \\ \hline
      $\lceil \log_2 x \rceil$ & \texttt{ilog2(x - 1)}: $x-1$ の 2 進数表現で最も左にある 1 の位置。 \\ \hline
      $\mathrm{popcount}(x)$ & \texttt{popcnt(x)}: $x$ のビット内に存在する 1 の個数。 \\ \hline
      $\mathrm{ctz}(x)$& \texttt{ctz(x)}: $x$ の最右ビットから連続する 0 の個数。count trailing zeros.\\ \hline
      $\mathrm{clz} = 
        \begin{cases}
        n - 1 - \lfloor \log_2 x \rfloor & \text{if } x > 0 \\
        n & \text{if } x = 0
        \end{cases}$ & \texttt{clz(x)}: $x$ の最左ビットから連続する 0 の個数。count leading zeros.\\ \hline
      $\mathrm{clo}(x)$ & \texttt{clz(\^x)}: $x$ の最左ビットから連続する 1 の個数。count leading ones.\\ \hline
  \end{tabular}
  \caption{数式表記からビット演算と CPU 命令への変換。}
  \label{table:cpu_instruction}
\end{table}

実際の演算においては、例えば $i$ を 64-bit 整数と仮定した場合に $j$ は $0 \le j \le 64$ の範囲を取ることができるため、$j=64$ のケースでオーバーフローを起こす可能性がある。この境界値チェックのために \texttt{if(j < 64)} のような条件分岐を追加する必要があるが、現実的な使用範囲では \texttt{j < 64} は常に \texttt{true} となり、CPU の分岐予測と投機実行に高い予測成功率を期待できるため、追加の境界チェックのオーバーヘッドはほぼ無視できると考えられる。

% \section{擬似コード}
% \section{証明補足}
% \section{評価スクリプトとデータセット概要}

\end{document}
