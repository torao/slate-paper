\documentclass[a4paper,11pt]{article}

\usepackage{geometry}
\geometry{margin=25mm}
\usepackage{titlesec}
\usepackage{setspace}
\usepackage{hyperref}
\usepackage{pxjahyper} % 日本語対応のハイパーリンク
\usepackage{indentfirst}
\setstretch{1.2}

\title{Slate: 追記最適化型ハッシュツリーによる構造的同期アルゴリズム}
\author{鷹見 虎男}
\date{\today}

\begin{document}
\maketitle

\begin{abstract}
% 要約をここに記述
\end{abstract}

\tableofcontents
\newpage

\section{はじめに}
\subsection{背景:ハッシュツリーの応用と課題}
\subsection{追記最適化の必要性}
\subsection{本研究の目的と貢献}
\subsection{本論文の構成}

\section{既存研究と位置づけ}
\subsection{Merkle Tree とその変種}
\subsection{ブロックチェーン・分散ログ・同期アルゴリズムへの応用}
\subsection{既存手法の問題点:対称性と時間的局所性の欠如}
\subsection{本研究の新規性}

\section{Stratified Hash Tree の設計原理}
\subsection{構造的アルゴリズムの概念}
\subsection{層化 (stratification) と追記の形式的定義}
\subsection{ノード配置とハッシュ計算規則}
\subsection{Slate の生成過程と更新手順}
\subsection{計算量と記憶効率の理論的解析}

\section{距離分布と探索特性}
\subsection{最新エントリからのハミング距離の定義}
\subsection{距離分布の二項分布性の証明}
\subsection{時間的局所性最適化 (Temporal Locality Optimization)}
\subsection{I/O 効率とキャッシュヒット率のモデル化}

\section{同期アルゴリズム}
\subsection{2 系統のデータ列間での差分検出}
\subsection{層化構造を用いた差分検出の計算手順}
\subsection{ネットワーク同期と部分ツリー転送の最適化}
\subsection{アルゴリズムの正当性と停止性の証明}

\section{実装と評価}
\subsection{実装概要(Rust/Scala 実装例)}
\subsection{実験設定とデータセット}
\subsection{パフォーマンス比較(Merkle Tree, LSM-Tree との比較)}
\subsection{追記性能・同期速度・メモリ消費の評価}
\subsection{考察:非対称性の実証}

\section{応用可能性と拡張}
\subsection{分散トランザクションログへの応用}
\subsection{認証付きデータ構造(ADS)への展開}
\subsection{キャッシュシステムへの応用}
\subsection{並列化・ストレージ最適化の方向性}

\section{関連理論との接続}
\subsection{情報理論・符号理論的観点}
\subsection{木構造データベースと部分順序理論の関係}
\subsection{確率的データ構造との比較}
\subsection{「構造的アルゴリズム」としての一般化}

\section{結論と今後の課題}
\subsection{本研究のまとめ}
\subsection{理論的拡張:確率解析と漸近挙動}
\subsection{応用拡張:分散同期・検証アルゴリズム}
\subsection{今後の研究課題}

\section*{参考文献}

\appendix
\section{擬似コード}
\section{証明補足}
\section{評価スクリプトとデータセット概要}

\end{document}
